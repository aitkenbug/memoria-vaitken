\chapter{Introducción}
\section{Motivacion}

La construccion de un nuevo instrumento de observacion astronomica conlleva diversos desafios, oportunidades y nuevos conocimientos. Por lo que el comisionamientio un nuevo radio telescopio de 3 metros de diametro es un proyecto que involucra distintos aspectos mecanicos, electronicos, de radio frecuencia o (RF) y de software. Para asegurar un funcionamiento correcto y poder realizar observaciones astronomicas para aportar a nuestros astroinomos y a la comunidad cientifica.\\

En la cumbre del Cerro Calan, en la ciudad de Santiago, se encuentra el Observatorio Astronomico Nacional y el departamento de astronomia de la facultad de ciencias fisicas y matematicas de la Universidad de Chile. Aqui se encuentra el telescopio CPT (CHARTS Pathfinder Telescope), un radio telescopio de reflector parabolico de 3 metros de diametro, de superficie de malla metalica y con una montura alt azimutal. En este documento se detalla el proceso de construccion mecanica, electronica y de software para la caracterización y puesta en servicio de este telescopio.\\ 

Las capacidades de observacion de un telescopio de radio son definidas con las caracteristicas de su antena receptora y sus propiedades de sensibilidad, resolucion angular y ancho de banda. Lo cual es tambien definido por el proposito de construccion y los intereses cientificos de los invwestigadores. Para el caso del CPT, se busca observar la linea de emision de hidrogeno neutro, a una frecuencia de 1420MHz, con el proposito de validar la funcionalidad del telescopio para realizar mediciones de radioastronomia. Ya que la emision de hidrogeno neutro o H1 es una de las mas fuertes en el espectro de radio y es una de las más estudiadas.\\

Luego de la puesta en servicio del telescopio, se espera poder adaptar el receptor para un ancho de banda superior y realizar estudios de interferometria de larga base con otros telescopios, estudiar la deteccion temprana de llamaradas solares y apoyar al proyecto CHARTS (Canadian-Chilean Array for Radio Transient Studies) en el estudio del fenomeno astrofisico de rafagas rapidas de radio (FRB).Por lo que todas las desiciones de diseño y construccion, fueron tomadas teniendo en cuentas todas las capacidades que se pensaron en la concepcion de este instrumento.\\

Durante este trabajo se detallan los procesos de construccion mecanica del reflector parabolico, el ensamblaje de la monstura alt azimutal, el diseño y construccion del receptor de radio frecuencia. Tambien se detallan los trabajos de caracterizacion del telescopio con la medicion de su patron de radiacion, su sensibilidad y resolucion angular. Para culminar con su primera luz, un evento en el cual todo telescopio observa por primera vez una fuente astronomica, dando al inicio a su vida util como instrumento de observacion.\\


\section{Objetivo General}
Las motivaciones que inspiraron la memoria incluyen la ayuda a la comunidad científica en los estudios cosmológicos y de radioastronomía, entregar nuevas herramientas para la investigación de los astrónomos de la universidad y los atractivos bajos costos que conllevan el diseño a construir.
\section{Objetivos Especificos}
\subsection{Ensamblado Mecánico}

Completar el ensamblado del reflector parabólico a utilizar, asegurando su integridad mecánica y funcionamiento del motor en su montura.

\subsection{Diseño del receptor}

Evaluación de las distintas opciones de diseño de la antena receptora y construir la electrónica de adquisición para las frecuencias de interés.

\subsection{Primera Luz y Caracterización}

Lograr caracterizar los parámetros de funcionamiento del telescopio y detectar las primeras ondas de radio.

\subsection{Observar Hidrógeno neutro}


Iniciar los estudios astronómicos observando una de las bandas de diseño de 1420MHz o línea de Hidrógeno.