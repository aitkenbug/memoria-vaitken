\chapter{Introducción}
\section{Motivación}

La construcción de un nuevo instrumento de observación astronómica conlleva diversos desafíos, oportunidades y nuevos conocimientos. El posicionamiento de un nuevo radiotelescopio de 3 metros de diámetro es un proyecto que involucra distintos aspectos mecánicos, electrónicos, de radiofrecuencia o RF y de software.\\

% Para asegurar un funcionamiento correcto y poder realizar observaciones astronómicas para aportar a nuestros astrónomos y a la comunidad científica.\\ REVISAR

En la cumbre del Cerro Calan, en la ciudad de Santiago, se encuentra el Observatorio Astronómico Nacional y el departamento de astronomía de la facultad de ciencias físicas y matemáticas de la Universidad de Chile. Aquí se encuentra el telescopio CPT (CHARTS Pathfinder Telescope), un radiotelescopio de reflector parabólico de 3 metros de diámetro, de superficie de malla metálica y con una montura alt azimutal. En este documento se detalla el proceso de construcción mecánica, electrónica y de software para la caracterización y puesta en servicio de este telescopio.\\ 

Las capacidades de observación de un telescopio de radio son definidas con las características de su antena receptora y sus propiedades de sensibilidad, resolución angular y ancho de banda. Lo cual es también definido por el propósito de construcción y los intereses científicos de los investigadores. Para el caso del CPT, se busca observar la línea de emisión de hidrógeno neutro, a una frecuencia de 1420 MHz, con el propósito de validar la funcionalidad del telescopio para realizar mediciones de radioastronomía. Ya que la emisión de hidrógeno neutro o H1 es una de las más estudiadas del espectro de radio.\\

Luego de la puesta en servicio del telescopio, se espera poder adaptar el receptor para un ancho de banda superior y realizar estudios de interferometría de larga base con otros telescopios, estudiar la detección temprana de llamaradas solares y apoyar al proyecto CHARTS (Canadian-Chilean Array for Radio Transient Studies) en el estudio del fenómeno astrofísico de ráfagas rápidas de radio (FRB). Por lo que todas las decisiones de diseño y construcción, fueron tomadas, teniendo en cuentas todas las capacidades que se pensaron en la concepción de este instrumento.\\

Durante este trabajo se detallan los procesos de construcción mecánica del reflector parabólico, el ensamblaje de la montura alt azimutal, el diseño y construcción del receptor de radiofrecuencia. También se detallan los trabajos de caracterización del telescopio con la medición de su patrón de radiación, su sensibilidad y resolución angular. Para culminar con su primera luz, un evento en el cual todo telescopio observa por primera vez una fuente astronómica, dando al inicio a su vida útil como instrumento de observación.\\


\section{Objetivo General}
Ensamblado, integracion y puesta en servicio de un radiotelescopio de 3 metros de diametro para docencia e investigacion en nuevas tecnologias para radioastronomía.

\section{Objetivos Específicos}
\subsection{Ensamblado Mecánico}

Ensamblar reflector parabólico a utilizar, asegurando su integridad mecánica y funcionamiento del motor en su montura.

\subsection{Diseño del receptor}

Evaluación de las distintas opciones de diseño de la antena receptora e integrar la electrónica de adquisición para las frecuencias de interés.

\subsection{Primera Luz y Caracterización}

Caracterizar los parámetros de funcionamiento del telescopio y detectar las primeras ondas de radio.

\subsection{Observar Hidrógeno neutro}


Iniciar los estudios astronómicos observando una de las bandas de diseño de 1420 MHz o línea de Hidrógeno.