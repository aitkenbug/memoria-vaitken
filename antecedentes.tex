\chapter{Antecedentes}
\section{Conceptos previos}
\subsection{Radiotelescopio}

En la astronomía clásica, se utilizan telescopios que operan en el rango visible del espectro electromagnético, el mismo rango que tiene el ojo humano, donde estos actúan como el receptor o, al mismo tiempo, una cámara fotográfica. En contraste, un radiotelescopio es un instrumento que observa en longitudes de onda mucho más grandes en el espectro de radio. Estos telescopios, usualmente, son constituidos por un reflector parabólico que concentra la luz obtenida en su foco, donde se ubica un receptor de radio.\\

Un telescopio de radio tiene consideraciones distintas a uno óptico, ya que el anterior puede observar con múltiples receptores a la vez como lo es el caso de una cámara fotográfica, pero uno de radio, debe tener un receptor de mayor envergadura proporcional con el tamaño de la longitud de onda, lo que limita el número de receptores a uno en la gran mayoría de los casos.\\

Otro punto importante a destacar, es la característica de ciencia que se puede realizar con estos telescopios, ya que para el caso de la línea de Hidrógeno, se pueden observar fuentes que no necesariamente provienen de estrellas o, para frecuencias más bajas, estrellas con que irradian fuera del rango visible.

\subsection{Línea de Hidrógeno Neutro}

El movimiento de un electrón en un átomo de Hidrógeno neutro, genera un campo magnético que se acopla con los espines del protón y el electrón. \quotes{Este acople da cuenta de la radiación a 1420 MHz que viene de la transición entre dos niveles energéticos de primer nivel del estado fundamental del hidrógeno} \cite{Restrepo2023}.\\

Observar H1 permite estudiar la evolución del universo primitivo, rescatando información proveniente de la transición de la \quotes{Época Oscura} a la formación de las primeras fuentes de luminosas del universo.

\subsection{CHARTS y FRB}

Los fenómenos astrofísicos transitorios de radio o FRB, son eventos de extremadamente corta duración y origen desconocido que ocurren en un amplio rango de frecuencias. Estos pulsos inspiraron el proyecto CHARTS, para apoyar su búsqueda y estudio.\\

El proyecto CHARTS, es una colaboración entre la Universidad de Chile y la Universidad de Toronto con el objetivo de construir un arreglo de 128 sintonizadas para operar en el rango de 300MHz a 500MHz en el marco de la búsqueda de FRB.
\section{Marco teórico}
A continuación, se explicarán los conceptos, herramientas y fundamentos para sustentar la construcción y operación de un nuevo radiotelescopio. Incluyendo los elementos teóricos más relevantes para realizar este proyecto.

\subsection{Patrón de Radiación}

El patrón de radiación es el gráfico de la potencia transmitida por la antena, evaluada sobre una esfera de radio constante\cite{Astudillo2014}. Usualmente, se estudian cortes del patrón que corresponden a las curvas en tres dimensiones de este, que son contenidas en la intersección de esta esfera pasando por el origen.\\

La medición de patrón de radiación, es la potencia que se obtiene en la aproximación de campo lejano. por lo que este es independiente de la distancia. El patrón de radiación, se define como el campo eléctrico y la potencia normalizada para ser expresada en decibelios que se representa por las siguientes ecuaciones.

\begin{equation}
    \Vec{F}(\theta, \Phi)=\frac{\Vec{E(\theta, \phi)}}{max|\Vec{E}(\theta, \phi)|}
\end{equation}
\begin{equation}
    P(\Theta, \Phi) = |\Vec{F}(\theta, \Phi)|^{2}
\end{equation}
\begin{equation}
    P(\Theta, \Phi)_{dB} = 10logP(\Theta, \Phi)=20log |\Vec{F}|=F(\Theta, \Phi)
\end{equation}

\begin{figure}
    \centering
    \includegraphics[width = 11cm]{img/patern.png}
    \caption{Patrón de radiación del telescopio Mini en el cerro Tololo con cortes en azimut y elevación\cite{Astudillo2014}.}
    \label{fig:patern}
\end{figure}

A partir de patrón de radiación, como el de la figura \ref{fig:patern}, se pueden obtener características muy relevantes para la caracterización y estudio de una antena y, particularmente, un radiotelescopio. Entre estas características, se encuentran la medida del ancho de haz de media potencia, o HPBM por su sigla en inglés.\\

También, se encuentra la radiactividad, que representa la concentración de potencia en un solo punto, definida como la razón entre el haz de mayor potencia de la antena estudiada y el haz de la antena isotrópica. Con esta medida, se define la ganancia y la eficiencia de antena, que se obtienen a partir de la potencia enfocada en la dirección deseada por el haz principal del patrón de radiación.\\

\begin{figure}
    \centering
    \includegraphics[width = 15cm]{img/patern2.png}
    \caption{Patrón de radiación teórico que compara la radiación isotrópica con la radiación de una antena directiva con la medida de haz de media potencia y la existencia de lóbulos laterales \cite{Cassanelli2022}.}
    \label{fig:patern2}
\end{figure}

La figura \ref{fig:patern2} representa un patrón de radiación producido por una antena altamente directiva ubicada en el centro de la circunferencia punteada. Esta antena esta orientada para que la concentración de radiación sea hacia la derecha de la figura.

\subsection{Polarización}

La polarización de una antena hace referencia a la polarización del campo eléctrico y magnético que produce al irradiar potencia al medio. Convencionalmente, se utilizan 2 polarizaciones para las observaciones astronómicas y para las telecomunicaciones, la polarización lineal y la polarización circular.\\

\begin{figure}
    \centering
    \includegraphics[width = 15cm]{img/pol.png}
    \caption{A la izquierda se tiene una polarización lineal del campo electromagnético y a la derecha una polarización circular izquierda\cite{Astudillo2014}.}
    \label{fig:pol}
\end{figure}

La polarización linear, se genera con el campo eléctrico contenido en un plano, como se puede ver en la figura \ref{fig:pol}. En contraste, la polarización circular se produce en un caso particular cuando el campo eléctrico rota con una frecuencia constante en torno a un eje y su magnitud es constante. De lo contrario, se produce una polarización elíptica en torno al eje de propagación, tal como se observa en la figura anterior.\\

Las distintas polarizaciones, se pueden generar por medios constructivos en la geometría de la antena o medios de alteración de fase, ya sea por efectos de largo eléctrico o modulaciones electrónicas para generar el desfase de las líneas alimentadoras.

\subsection{Antenas de Apertura}

Las antenas de apertura son aquellas que su funcionamiento es caracterizado por el campo generado en una superficie reflectante, la cual se denomina como apertura. Dentro de las antenas de apertura están las antenas de apertura parabólica que utilizan una superficie parabólica para irradiar potencia. Existen 4 tipos de configuraciones para las antenas parabólicas, \textit{Cassegrain}, \textit{Gregorian}, \textit{off-axis} o fuera de foco y \textit{axial feed} o Foco Primario. Esta última es el que se utilizará para el proyecto.\\

Estas antenas concentran la radiación en un punto focal dependiendo de su geometría de paraboloide. Usualmente, se instala la antena alimentadora en el foco de la parábola o en el foco de la segunda parábola para las antenas \textit{Cassegrain} y \textit{Gregorian}.\\

\begin{figure}
    \centering
    \includegraphics[width = 10cm]{img/parabola.png}
    \caption{Geometría característica de un reflector parabólico para el tipo de Foco Primario\cite{Astudillo2014}.}
    \label{fig:parabola}
\end{figure}

La representación de seguimiento de rayos en dos dimensiones para un reflector parabólico en la figura \ref{fig:parabola}, hace referencia al tipo de configuración a utilizar en la construcción del reflector del nuevo radio telescopio.


\subsection{Antena Alimentadora}

La antena alimentadora, es aquella antena que captura la radiación proveniente de los reflectores. Se debe ocupar una antena con un patrón de radiación directivo, con el objetivo de captar la mayor cantidad de la luz concentrada por el reflector.\\

Ejemplos de antenas directivas:

\begin{itemize}
    \item Yagi-Uda
    \item Bocina
    \item Log-Periódica
    \item Antenas de semi espacio
    \item Antenas de parche
\end{itemize}

\subsection{Receptor Heterodino}

Los receptores heterodinos, o coherentes, son los más usados en la radioastronomía. Su función característica es convertir una señal de alta frecuencia a un rango de menor frecuencia, conservando la información de fase y de amplitud para poder ser digitalizada con facilidad\cite{Finger2013}. Para esto, los receptores utilizan un mezclador con un oscilador local para adquirir la señal desde la antena alimentadora para luego ser procesada digitalmente.\\

\begin{figure}
    \centering
    \includegraphics[width = 10cm]{img/heterodine.png}
    \caption{Diagrama de bloques de un receptor heterodino para astronomía de tipo DSB.}
    \label{fig:radio}
\end{figure}

La figura \ref{fig:radio} corresponde a un diagrama de bloques de la configuracion más simple para un receptor heterodino. Contiene una antena receptora, un mezclador de radio frecuencia, un oscilador local y un amplificador de bajo ruido.

\section{Estado del Arte}

El estado del arte para radiotelescopios, en particular de radios telescopios de reflectores con diámetros de 3 metros o menos. En este caso, la principal línea de desarrollo es en la radio-interferometría y el uso de sistemas de bajo ruido, a la vez que la disminución de los costos para nuevos instrumentos.\\

Los esfuerzos para la búsqueda de FRB y, específicamente, en la detección del origen de estos pulsos de alta energía, puede lograrse con el uso de \textit{Pathfinding} al correlacionar 2 telescopios separados por largas distancias para hacer interferometría de línea de base muy larga, o VLBI por su sigla en inglés (T. A. Cassanelli, 2021). Experimento que se quiere hacer con el proyecto CHARTS y este nuevo telescopio.\\

Luego, en el contexto de la construcción un nuevo radiotelescopio para observar la línea de 21cm, se tienen los siguientes exponentes que implementan diversas técnicas que inspiran el desarrollo y operación que se desea con el telescopio CPT.\\


\subsection{Telescopio Mini}

\begin{figure}
    \centering
    \includegraphics[width = 12cm]{img/mini.png}
    \caption{Telescopio MINI en el Observatorio Cerro Calan de la Facultad de Ciencia Físicas y Matemáticas de la Universidad de Chile\cite{DAS2024}}
    \label{fig:mini}
\end{figure}

El telescopio de la figura \ref{fig:mini}, se encuentra en el cerro Calán, a pocos metros de la ubicación de los cimientos para, CPT, ejemplificando las consideraciones de operar en un ambiente extremadamente saturado de interferencia de radiofrecuencia que se puede encontrar en el centro de la ciudad principal del país como Santiago.

\subsection{Telescopio FAST}

\begin{figure}
    \centering
    \includegraphics[width = 12cm]{img/fast.png}
    \caption{Telescopio Fast en la provincia de Guizhou, China \cite{FAST2024}.}
    \label{fig:fast}
\end{figure}

La figura \ref{fig:fast}, corresponde a un telescopio de 500 metros de apertura con un alimentador posicionado en el foco de la superficie parabólica primaria que puede hacer observaciones de la linea de 21cm y de FRB. El cual se ubica en China, en la provincia de Guizhou.

\subsection{Telescopio SRT}

\begin{figure}
    \centering
    \includegraphics[width = 6cm]{img/srt.png}
    \caption{Telescopio SRT del MIT en la azotea del departamento de ingeniería eléctrica de la FCFM\cite{Curotto2019}.}
    \label{fig:srt}
\end{figure}

El Telescopio de la figura \ref{fig:srt}, se encuentra ubicado en la azotea del edificio del Departamento Ingeniería Eléctrica de la Universidad de Chile, equipado con receptores y filtros diseñados para observar la línea de Hidrógeno neutro. El material de sus reflectores es similar al que se utilizará en CPT.

\subsection{Observatorio CHIME}

\begin{figure}
    \centering
    \includegraphics[width = 12cm]{img/chime.png}
    \caption{Telescopio experimental CHIME en Canada \cite{CHIME}.}
    \label{fig:chime}
\end{figure}

La figura \ref{fig:chime}, corresponde al telescopio experimental sin partes móviles, sintonizado para observar Hidrógeno y con un correlacionador capaz generar una apertura sintética para encontrar fenómenos astrofísicos transitorios de radio. Trabaja con una arquitectura de instrumentos remotos.

\subsection{Guía de Construcción de Radiotelescopio con una RTL-SDR}

Un enfoque para hacer radioastronomía con componentes comerciales y de bajo costo. Se utilizan radios definidas por software como receptor de radiofrecuencia y destaca las herramientas y software para observar la línea de 21cm.

\begin{figure}
    \centering
    \includegraphics[width = 12cm]{img/rtlsdr.png}
    \caption{Software utilizado por RTL-SDR Blog para observar Hidrógeno neutro con una RTL-SDR\cite{RTLSDR2018}.}
    \label{fig:rtl}
\end{figure}

La figura \ref{fig:rtl}, corresponde al software de recepción de radio sintonizado en la frecuencia de emisión del hidrógeno neutro a la derecha, además del software de seguimiento astronómico y catalogo de objetos de interés.