% Template:     Tesis LaTeX
% Documento:    Archivo principal
% Versión:      3.4.0 (23/08/2024)
% Codificación: UTF-8
%
% Autor: Pablo Pizarro R.
%        pablo@ppizarror.com
%
% Manual template: [https://latex.ppizarror.com/tesis]
% Licencia MIT:    [https://opensource.org/licenses/MIT]

% CREACIÓN DEL DOCUMENTO
\documentclass[
	spanish, % Idioma: spanish, english, etc.	
	letterpaper, oneside
]{book}

% INFORMACIÓN DEL DOCUMENTO
\def\documenttitle {Puesta en servicio del telescopio CPT (CHARTS Pathfinder Telescope) en cerro Calán}
\def\documentsubtitle {}
\def\degreetitle {
	Memoria para optar al título de ingeniero civil eléctrico
}

\def\universityname {Universidad de Chile}
\def\universityfaculty {Facultad de Ciencias Físicas y Matemáticas}
\def\universitydepartment {Departamento de Ingeniería Eléctrica}
\def\universitydepartmentimage {departamentos/uchile2}
\def\universitydepartmentimagecfg {height=3cm}
\def\universitylocation {Santiago de Chile}

% INTEGRANTES, PROFESORES Y FECHAS
\def\documentauthor {Vicente Rodrigo Aitken Albornoz}
\def\documentdate {\the\year}

\def\portrait {
	\begin{center}
	\vspace{1.5cm} ~ \\
	\MakeUppercase{\textbf{\documenttitle}} ~ \\
	\vspace{1.5cm}
	\MakeUppercase{\degreetitle} ~ \\
	\vfill
	\begin{tabular}{c}
		\MakeUppercase{\textbf{\documentauthor}} \\ \\
		\vspace{1.0cm} \\
		PROFESOR GUÍA: \\
		RICARDO FINGER CAMUS\\
		\vspace{0.5cm} \\
		MIEMBROS DE LA COMISIÓN: \\
		TOMAS CASSANELLI ESPEJO\\
		FRANCO CUROTTO MOLINA\\
		\vspace{0.5cm} \\
		Este trabajo ha sido financiado por: \\
		Fondo ANID Basal FB210003, FONDEF ID21-10359.\\
		 Fondecyt 1221662, ANID / Fondo 2023 QUIMAL / QUIMAL230001.\\
		  Dunlap Seed Founding Program por la familia David Dunlap y la Universidad de Toronto.\\

		\vspace{0.5cm} \\
		\MakeUppercase{\universitylocation} \\
		\MakeUppercase{\documentdate}
	\end{tabular}
	\end{center}
}
\def\abstracttable {
	\begin{tabular}{l}
		RESUMEN DE LA MEMORIA PARA OPTAR \\
		AL TÍTULO DE INGENIERO CIVIL ELÉCTRICO \\
		POR: \MakeUppercase{\documentauthor} \\
		FECHA: \MakeUppercase{\documentdate} \\
		PROF. GUÍA: RICARDO FIGNER CAMUS
	\end{tabular}
}

% IMPORTACIÓN DEL TEMPLATE
\input{template}

% INICIO DE LAS PÁGINAS
\begin{document}

% PORTADA
\templatePortrait

% CONFIGURACIÓN DE PÁGINA Y ENCABEZADOS
\templatePagecfg

% RESUMEN O ABSTRACT
\begin{abstractd}
	En el presente trabajo se describe el ensamblaje, caracterización y puesta en servicio del nuevo telescopio \textit{CHARTS Pathfinder Telescope} (CPT), un radiotelescopio de 3 metros de diámetro ubicado en la cumbre del cerro Calán, en el Observatorio Astronómico Nacional de Chile. Este proyecto tiene un enfoque práctico en el desarrollo de herramientas para la astronomía, involucrando diciplinas como la mecánica, electrónica, radiofrecuencia y software.\\
	
	El proceso de ensamblaje significó el montaje del reflector parabólico, la montura alt-azimutal y el receptor de radiofrecuencia. El sistema electrónico y de adquisición se centró en una solución de bajo costo y alta eficiencia, utilizando radios definidas por software, amplificadores de bajo ruido y filtros de pasabanda. También se diseñaron softwares para la observación de cuerpos celestes y calibración del instrumento.\\

	La caracterización del telescopio incluyó la medición del patrón de radiación para sus frecuencias de interés, se midió un haz de media potencia de 4.5 grados a 1.4 GHz y 7.5 a 400 MHz donde se obtuvo una ganancia de 31.5 dBi y una directividad (dB) de 39.7 para la banda de calibración de 21 cm de longitud de onda conocida como HI. Los resultados son consistentes con la teoría para una antena parabolica de 3 metros con una eficiencia de XX a 400 MHz y 63\% a 1.42 GHz. El proyecto culmina con la \quotes{primera luz} al observar la emisión de H1 en el centro Galáctico. La observación es consistente con la emisión reportada para región estandar S9 de calibración\\

	Con la operación del CPT se espera poder realizar actividades de docencia e investigación, como interferometría de larga base, estudios de llamaradas solares y apoyo  en el estudio de ráfagas rápidas de radio. Este trabajo significa un avance en la capacidad técnica de instrumentación astronómica para la comunidad científica local así como también un recurso para colaboraciones internacionales como la iniciativa del proyecto CHARTS. \\
\end{abstractd}

% DEDICATORIA
\begin{dedicatory}
	But remember this, Japanese boy...\\
	airplanes are not tools for war.\\
	They are not for making money.\\ 
	Airplanes are beautiful dreams. \\
	Engineers turn dreams into reality.\\

	\textbf{- Hayao Miyazaki}
\end{dedicatory}

% AGRADECIMIENTOS
\begin{acknowledgments}
	Primero que todo quiero agradecer a mi profesor guía Ricardo Finger, quien no solo me entregó esta oportunidad de liderar un proyecto de tal envergadura, sino que también entregó importantes esfuerzos para llevarlo adelante y me devolvió la oportunidad de realizar algo que en algún momento solo era un sueño. Agradezco a Tomás Cassanelli por creer en mis habilidades y permitirme involucrarme en el proyecto y a Franco Curotto, quien me entrego valiosos consejos y apoyo en el proceso y nunca dudó del éxito que alcanzaría.\\

	Agradezco a mis segundas familias, quienes sin su apoyo probablemente no haya llegado tan lejos. Gracias equipo del laboratorio MWL, Claudia y el humor, Gonzalo y el conocimiento, Diego y las pesas, Juan y los chistes, José y las piezas, Sebita, Juan Francisco y el curso, Pancho, José, gracias por su apoyo en este proyecto, la calidez y el humor que levanta la frente cada lunes.\\

	Gracias equipo de atletismo de la FCFM y de la Universidad de Chile, por ser mi fuente de energía y motivación en los momentos más difíciles y por ser mi escape de la rutina. Gracias Joaco, Karen, Profesor Mario, Fica, Fefi, Chimi, lanzadores, que la pista y el foso no sería lo mismo sin ustedes.\\

	Gracias a mi familia por su apoyo incondicional, por creer en mí y darme todo para poder avanzar en este viaje tan largo y lleno de desafíos. Gracias, amigos, por estar siempre ahí, aunque los años pasen y las distancias se hagan más grandes, siempre están en mi corazón.\\

	Gracias a los primeros, Ro y Joaco, los que me siguieron, Pepe, Jano y Yohans, y los encontré después, Cata, Magda, Diego, Marcelo, Cami, Mati. Sinceramente, la mejor época de nuestras vidas, no hubiera sido lo mismo sin ustedes y muchos otros que no nombro.\\

	Dejo una mención a esas bandas que me acompañaron en los momentos de estudio, de trabajo, de relajo, de fiesta, de tristeza, de felicidad, de amor, de desamor, de todo. Muse, System of A Down, Alter Bridge, The Wombats, Radiohead, Jorge Drexler entre muchas otras.\\
\end{acknowledgments}

% TABLA DE CONTENIDOS - ÍNDICE
\templateIndex

% CONFIGURACIONES FINALES
\templateFinalcfg

% ======================= INICIO DEL DOCUMENTO =======================

\chapter{Introducción}
\section{Motivación}

La construcción de un nuevo instrumento de observación astronómica conlleva diversos desafíos, oportunidades y nuevos conocimientos. Por lo que el posicionamiento un nuevo radiotelescopio de 3 metros de diámetro es un proyecto que involucra distintos aspectos mecánicos, electrónicos, de radiofrecuencia o RF y de software. Para asegurar un funcionamiento correcto y poder realizar observaciones astronómicas para aportar a nuestros astrónomos y a la comunidad científica.\\

En la cumbre del Cerro Calan, en la ciudad de Santiago, se encuentra el Observatorio Astronómico Nacional y el departamento de astronomía de la facultad de ciencias físicas y matemáticas de la Universidad de Chile. Aquí se encuentra el telescopio CPT (CHARTS Pathfinder Telescope), un radiotelescopio de reflector parabólico de 3 metros de diámetro, de superficie de malla metálica y con una montura alt azimutal. En este documento se detalla el proceso de construcción mecánica, electrónica y de software para la caracterización y puesta en servicio de este telescopio.\\ 

Las capacidades de observación de un telescopio de radio son definidas con las características de su antena receptora y sus propiedades de sensibilidad, resolución angular y ancho de banda. Lo cual es también definido por el propósito de construcción y los intereses científicos de los investigadores. Para el caso del CPT, se busca observar la línea de emisión de hidrógeno neutro, a una frecuencia de 1420 MHz, con el propósito de validar la funcionalidad del telescopio para realizar mediciones de radioastronomía. Ya que la emisión de hidrógeno neutro o H1 es una de las más fuertes en el espectro de radio y es una de las más estudiadas.\\

Luego de la puesta en servicio del telescopio, se espera poder adaptar el receptor para un ancho de banda superior y realizar estudios de interferometría de larga base con otros telescopios, estudiar la detección temprana de llamaradas solares y apoyar al proyecto CHARTS (Canadian-Chilean Array for Radio Transient Studies) en el estudio del fenómeno astrofísico de ráfagas rápidas de radio (FRB). Por lo que todas las decisiones de diseño y construcción, fueron tomadas, teniendo en cuentas todas las capacidades que se pensaron en la concepción de este instrumento.\\

Durante este trabajo se detallan los procesos de construcción mecánica del reflector parabólico, el ensamblaje de la montura alt azimutal, el diseño y construcción del receptor de radiofrecuencia. También se detallan los trabajos de caracterización del telescopio con la medición de su patrón de radiación, su sensibilidad y resolución angular. Para culminar con su primera luz, un evento en el cual todo telescopio observa por primera vez una fuente astronómica, dando al inicio a su vida útil como instrumento de observación.\\


\section{Objetivo General}
Las motivaciones que inspiraron la memoria incluyen la ayuda a la comunidad científica en los estudios cosmológicos y de radioastronomía, entregar nuevas herramientas para la investigación de los astrónomos de la universidad y los atractivos bajos costos que conllevan el diseño a construir.
\section{Objetivos Específicos}
\subsection{Ensamblado Mecánico}

Completar el ensamblado del reflector parabólico a utilizar, asegurando su integridad mecánica y funcionamiento del motor en su montura.

\subsection{Diseño del receptor}

Evaluación de las distintas opciones de diseño de la antena receptora y construir la electrónica de adquisición para las frecuencias de interés.

\subsection{Primera Luz y Caracterización}

Lograr caracterizar los parámetros de funcionamiento del telescopio y detectar las primeras ondas de radio.

\subsection{Observar Hidrógeno neutro}


Iniciar los estudios astronómicos observando una de las bandas de diseño de 1420 MHz o línea de Hidrógeno.
\chapter{Antecedentes}
\section{Conceptos previos}
\subsection{Radiotelescopio}

En la astronomía clásica, se utilizan telescopios que operan en el rango visible del espectro electromagnético, el mismo rango que tiene el ojo humano, donde estos actúan como el receptor o, al mismo tiempo, una cámara fotográfica. En contraste, un radiotelescopio es un instrumento que observa en longitudes de onda mucho más grandes en el espectro de radio. Estos telescopios, usualmente, son constituidos por un reflector parabólico que concentra la luz obtenida en su foco, donde se ubica un receptor de radio.\\

Un telescopio de radio tiene consideraciones distintas a uno óptico, ya que el anterior puede observar con múltiples receptores a la vez como lo es el caso de una cámara fotográfica, pero uno de radio, debe tener un receptor de mayor envergadura proporcional con el tamaño de la longitud de onda, lo que limita el número de receptores a uno en la gran mayoría de los casos.\\

Otro punto importante a destacar, es la característica de ciencia que se puede realizar con estos telescopios, ya que para el caso de la línea de Hidrógeno, se pueden observar fuentes que no necesariamente provienen de estrellas o, para frecuencias más bajas, estrellas con que irradian fuera del rango visible.

\subsection{Línea de Hidrógeno Neutro}

El movimiento de un electrón en un átomo de Hidrógeno neutro, genera un campo magnético que se acopla con los espines del protón y el electrón. \quotes{Este acople da cuenta de la radiación a 1420 MHz que viene de la transición entre dos niveles energéticos de primer nivel del estado fundamental del hidrógeno} \cite{Restrepo2023}.\\

Observar H1 permite estudiar la evolución del universo primitivo, rescatando información proveniente de la transición de la \quotes{Época Oscura} a la formación de las primeras fuentes de luminosas del universo.

\subsection{CHARTS y FRB}

Los fenómenos astrofísicos transitorios de radio o FRB, son eventos de extremadamente corta duración y origen desconocido que ocurren en un amplio rango de frecuencias. Estos pulsos inspiraron el proyecto CHARTS, para apoyar su búsqueda y estudio.\\

El proyecto CHARTS, es una colaboración entre la Universidad de Chile y la Universidad de Toronto con el objetivo de construir un arreglo de 128 sintonizadas para operar en el rango de 300MHz a 500MHz en el marco de la búsqueda de FRB.
\section{Marco teórico}
A continuación, se explicarán los conceptos, herramientas y fundamentos para sustentar la construcción y operación de un nuevo radiotelescopio. Incluyendo los elementos teóricos más relevantes para realizar este proyecto.

\subsection{Patrón de Radiación}

El patrón de radiación es el gráfico de la potencia transmitida por la antena, evaluada sobre una esfera de radio constante\cite{Astudillo2014}. Usualmente, se estudian cortes del patrón que corresponden a las curvas en tres dimensiones de este, que son contenidas en la intersección de esta esfera pasando por el origen.\\

La medición de patrón de radiación, es la potencia que se obtiene en la aproximación de campo lejano. por lo que este es independiente de la distancia. El patrón de radiación, se define como el campo eléctrico y la potencia normalizada para ser expresada en decibelios que se representa por las siguientes ecuaciones.

\begin{equation}
    \Vec{F}(\theta, \Phi)=\frac{\Vec{E(\theta, \phi)}}{max|\Vec{E}(\theta, \phi)|}
\end{equation}
\begin{equation}
    P(\Theta, \Phi) = |\Vec{F}(\theta, \Phi)|^{2}
\end{equation}
\begin{equation}
    P(\Theta, \Phi)_{dB} = 10logP(\Theta, \Phi)=20log |\Vec{F}|=F(\Theta, \Phi)
\end{equation}

\begin{figure}
    \centering
    \includegraphics[width = 11cm]{img/patern.png}
    \caption{Patrón de radiación del telescopio Mini en el cerro Tololo con cortes en azimut y elevación\cite{Astudillo2014}.}
    \label{fig:patern}
\end{figure}

A partir de patrón de radiación, como el de la figura \ref{fig:patern}, se pueden obtener características muy relevantes para la caracterización y estudio de una antena y, particularmente, un radiotelescopio. Entre estas características, se encuentran la medida del ancho de haz de media potencia, o HPBM por su sigla en inglés.\\

También, se encuentra la radiactividad, que representa la concentración de potencia en un solo punto, definida como la razón entre el haz de mayor potencia de la antena estudiada y el haz de la antena isotrópica. Con esta medida, se define la ganancia y la eficiencia de antena, que se obtienen a partir de la potencia enfocada en la dirección deseada por el haz principal del patrón de radiación.\\

\begin{figure}
    \centering
    \includegraphics[width = 15cm]{img/patern2.png}
    \caption{Patrón de radiación teórico que compara la radiación isotrópica con la radiación de una antena directiva con la medida de haz de media potencia y la existencia de lóbulos laterales \cite{Cassanelli2022}.}
    \label{fig:patern2}
\end{figure}

La figura \ref{fig:patern2} representa un patrón de radiación producido por una antena altamente directiva ubicada en el centro de la circunferencia punteada. Esta antena esta orientada para que la concentración de radiación sea hacia la derecha de la figura.

\subsection{Polarización}

La polarización de una antena hace referencia a la polarización del campo eléctrico y magnético que produce al irradiar potencia al medio. Convencionalmente, se utilizan 2 polarizaciones para las observaciones astronómicas y para las telecomunicaciones, la polarización lineal y la polarización circular.\\

\begin{figure}
    \centering
    \includegraphics[width = 15cm]{img/pol.png}
    \caption{A la izquierda se tiene una polarización lineal del campo electromagnético y a la derecha una polarización circular izquierda\cite{Astudillo2014}.}
    \label{fig:pol}
\end{figure}

La polarización linear, se genera con el campo eléctrico contenido en un plano, como se puede ver en la figura \ref{fig:pol}. En contraste, la polarización circular se produce en un caso particular cuando el campo eléctrico rota con una frecuencia constante en torno a un eje y su magnitud es constante. De lo contrario, se produce una polarización elíptica en torno al eje de propagación, tal como se observa en la figura anterior.\\

Las distintas polarizaciones, se pueden generar por medios constructivos en la geometría de la antena o medios de alteración de fase, ya sea por efectos de largo eléctrico o modulaciones electrónicas para generar el desfase de las líneas alimentadoras.

\subsection{Antenas de Apertura}

Las antenas de apertura son aquellas que su funcionamiento es caracterizado por el campo generado en una superficie reflectante, la cual se denomina como apertura. Dentro de las antenas de apertura están las antenas de apertura parabólica que utilizan una superficie parabólica para irradiar potencia. Existen 4 tipos de configuraciones para las antenas parabólicas, \textit{Cassegrain}, \textit{Gregorian}, \textit{off-axis} o fuera de foco y \textit{axial feed} o Foco Primario. Esta última es el que se utilizará para el proyecto.\\

Estas antenas concentran la radiación en un punto focal dependiendo de su geometría de paraboloide. Usualmente, se instala la antena alimentadora en el foco de la parábola o en el foco de la segunda parábola para las antenas \textit{Cassegrain} y \textit{Gregorian}.\\

\begin{figure}
    \centering
    \includegraphics[width = 10cm]{img/parabola.png}
    \caption{Geometría característica de un reflector parabólico para el tipo de Foco Primario\cite{Astudillo2014}.}
    \label{fig:parabola}
\end{figure}

La representación de seguimiento de rayos en dos dimensiones para un reflector parabólico en la figura \ref{fig:parabola}, hace referencia al tipo de configuración a utilizar en la construcción del reflector del nuevo radio telescopio.


\subsection{Antena Alimentadora}

La antena alimentadora, es aquella antena que captura la radiación proveniente de los reflectores. Se debe ocupar una antena con un patrón de radiación directivo, con el objetivo de captar la mayor cantidad de la luz concentrada por el reflector.\\

Ejemplos de antenas directivas:

\begin{itemize}
    \item Yagi-Uda
    \item Bocina
    \item Log-Periódica
    \item Antenas de semi espacio
    \item Antenas de parche
\end{itemize}

\subsection{Receptor Heterodino}

Los receptores heterodinos, o coherentes, son los más usados en la radioastronomía. Su función característica es convertir una señal de alta frecuencia a un rango de menor frecuencia, conservando la información de fase y de amplitud para poder ser digitalizada con facilidad\cite{Finger2013}. Para esto, los receptores utilizan un mezclador con un oscilador local para adquirir la señal desde la antena alimentadora para luego ser procesada digitalmente.\\

\begin{figure}
    \centering
    \includegraphics[width = 10cm]{img/heterodine.png}
    \caption{Diagrama de bloques de un receptor heterodino para astronomía de tipo DSB.}
    \label{fig:radio}
\end{figure}

La figura \ref{fig:radio} corresponde a un diagrama de bloques de la configuracion más simple para un receptor heterodino. Contiene una antena receptora, un mezclador de radio frecuencia, un oscilador local y un amplificador de bajo ruido.

\section{Estado del Arte}

El estado del arte para radiotelescopios, en particular de radios telescopios de reflectores con diámetros de 3 metros o menos. En este caso, la principal línea de desarrollo es en la radio-interferometría y el uso de sistemas de bajo ruido, a la vez que la disminución de los costos para nuevos instrumentos.\\

Los esfuerzos para la búsqueda de FRB y, específicamente, en la detección del origen de estos pulsos de alta energía, puede lograrse con el uso de \textit{Pathfinding} al correlacionar 2 telescopios separados por largas distancias para hacer interferometría de línea de base muy larga, o VLBI por su sigla en inglés (T. A. Cassanelli, 2021). Experimento que se quiere hacer con el proyecto CHARTS y este nuevo telescopio.\\

Luego, en el contexto de la construcción un nuevo radiotelescopio para observar la línea de 21cm, se tienen los siguientes exponentes que implementan diversas técnicas que inspiran el desarrollo y operación que se desea con el telescopio CPT.\\


\subsection{Telescopio Mini}

\begin{figure}
    \centering
    \includegraphics[width = 12cm]{img/mini.png}
    \caption{Telescopio MINI en el Observatorio Cerro Calan de la Facultad de Ciencia Físicas y Matemáticas de la Universidad de Chile\cite{DAS2024}}
    \label{fig:mini}
\end{figure}

El telescopio de la figura \ref{fig:mini}, se encuentra en el cerro Calán, a pocos metros de la ubicación de los cimientos para, CPT, ejemplificando las consideraciones de operar en un ambiente extremadamente saturado de interferencia de radiofrecuencia que se puede encontrar en el centro de la ciudad principal del país como Santiago.

\subsection{Telescopio FAST}

\begin{figure}
    \centering
    \includegraphics[width = 12cm]{img/fast.png}
    \caption{Telescopio Fast en la provincia de Guizhou, China \cite{FAST2024}.}
    \label{fig:fast}
\end{figure}

La figura \ref{fig:fast}, corresponde a un telescopio de 500 metros de apertura con un alimentador posicionado en el foco de la superficie parabólica primaria que puede hacer observaciones de la linea de 21cm y de FRB. El cual se ubica en China, en la provincia de Guizhou.

\subsection{Telescopio SRT}

\begin{figure}
    \centering
    \includegraphics[width = 6cm]{img/srt.png}
    \caption{Telescopio SRT del MIT en la azotea del departamento de ingeniería eléctrica de la FCFM\cite{Curotto2019}.}
    \label{fig:srt}
\end{figure}

El Telescopio de la figura \ref{fig:srt}, se encuentra ubicado en la azotea del edificio del Departamento Ingeniería Eléctrica de la Universidad de Chile, equipado con receptores y filtros diseñados para observar la línea de Hidrógeno neutro. El material de sus reflectores es similar al que se utilizará en CPT.

\subsection{Observatorio CHIME}

\begin{figure}
    \centering
    \includegraphics[width = 12cm]{img/chime.png}
    \caption{Telescopio experimental CHIME en Canada \cite{CHIME}.}
    \label{fig:chime}
\end{figure}

La figura \ref{fig:chime}, corresponde al telescopio experimental sin partes móviles, sintonizado para observar Hidrógeno y con un correlacionador capaz generar una apertura sintética para encontrar fenómenos astrofísicos transitorios de radio. Trabaja con una arquitectura de instrumentos remotos.

\subsection{Guía de Construcción de Radiotelescopio con una RTL-SDR}

Un enfoque para hacer radioastronomía con componentes comerciales y de bajo costo. Se utilizan radios definidas por software como receptor de radiofrecuencia y destaca las herramientas y software para observar la línea de 21cm.

\begin{figure}
    \centering
    \includegraphics[width = 12cm]{img/rtlsdr.png}
    \caption{Software utilizado por RTL-SDR Blog para observar Hidrógeno neutro con una RTL-SDR\cite{RTLSDR2018}.}
    \label{fig:rtl}
\end{figure}

La figura \ref{fig:rtl}, corresponde al software de recepción de radio sintonizado en la frecuencia de emisión del hidrógeno neutro a la derecha, además del software de seguimiento astronómico y catalogo de objetos de interés.
\chapter{Ensamblaje e instrumentacion}

En este capitulo se presentaran todos los detalles del ensamblado del reflector parabolico, la instalacion del rotor y la integracion de estos con el soporte de la montura en el pedestal construido para el telescopio. Tambien se detallaran todos los instrumentos evaluados y seleccionados para la construccion del receptor de radiofrecuencia, el rack de control y la infraestructura de caracterizacion.\\

Junto con esto, se mosntraran todas las piezas diseñadas e impresas en 3D para el soporte del alimentador y todos los soportes especificos que se necesitaron para la instalacion de los distintos componentes del telescopio.\\

Para finalizar con la descripcion del software creado para la operacion, mantenimiento y caracterizacion del telescopio.\\

\section{Ensamblado Mecánico}

Tanto el reflector parabolico como la montura alt azimutal y su correspondiente controlador, son elementos adquiridos de la compañia \textit{RFHamdesign}, una empresa holandesa que se especializa en la construccion de telescopios de radio aficionados. El reflector de 3 metros venia completamente desarmado y con piezas que requerian ser modificadas y ensambladas para su correcto funcionamiento.\\

Para todo el ensamblado se utilizaron herramientas de y electricas, como taladors, tijeras de ojalata, remachadoras, etc.\\

\begin{figure}
    \centering
    \includegraphics[width=0.8\textwidth]{img/herramientas}
    \caption{Herramientas utilizadas para el ensamblado de la superficie del reflector parabólico.}
    \label{fig:ensamblado1}
\end{figure}

En la figura \ref{fig:ensamblado1} se pueden ver las herramientas utilizadas para el ensamblado de la superficie del reflector parabólico ademas de las piezas que requerian de modificacion adicional para la instalacion correcta.\\

\subsection{Reflector Parabólico}

Las piezas del reflector se dividen en los 12 arcos, o costillas, de aluminio que conforman la estructura que da forma a la superficie parabolica, con un centro de aluminion donde estas 12 piezas se unen y apernan.\\

\begin{figure}
    \centering
    \includegraphics[width=0.8\textwidth]{img/estructura1}
    \caption{Los 12 arcos de alumunio apernados al centro del reflector parabólico.}
    \label{fig:ensamble2}
\end{figure}

En la figura \ref{fig:ensamble2} se pueden ver los 12 arcos de aluminio apernados a los discos de distribucion, que ademas es el punto de anclaje para el soporte de la montura.\\

Luego desenrollan y enderezan los tubos de aluminio que confirmar los anillos donde se tensaeran las mallas metalicas que conforman la superficie del reflector. Con la misma logica se toma la cinta de alumninio, que es aproximadamente de 4 mm de espesor, para enderesarla y prepara las perforaciones para los primeros remaches.\\

\begin{figure}
    \centering
    \includegraphics[width=0.8\textwidth]{img/estructura2}
    \caption{Los tubos de aluminio y la cinta de aluminio para la tensión de la malla metálica instalados radialmente en los soportes.}
    \label{fig:ensamble3}
\end{figure}

\subsection{Diseño de soportes adicionales}

Para poder instalar todos los componentes del telescopio, se debian fabricar soportes personalizados y adicionales para así poder utilizar receptores y elementos que no fueran parte del kit original del fabricante. Con el objetivo de reducir los timepos de fabricación y prototipado al usar componentes de aluminio o acero se decidio utilizar impresion 3D con filamento plastico PLA\footnote{Explicacion PLA} de alta resistencia mecanica.\\

Se diseñaron 6 piezas en total con el software de diseño asistido por computadora o \textit{CAD} \textit{Fusion 360} de la compañia \textit{Autodesk}. Todos los comonentes fueron impresos en PLA de alta resistencia o \textit{Hyper-PLA} de la compañia \textit{Creality}, otorgando una mayor resitencia a la flexion de 50\% que el PLA convencional y una elongacion de 6.304\% en comparacion con la del PLA convencional de 3\%. La configuracion de la impresion fue una altura de capa de 0.2 mm, dada por la boquilla utilizada, 4 capas de muralla y un \textit{Infill} o relleno de 60 \%.\\

Una ventaja importante en la elección de la impresion 3D en filamentops plasticos, es su baja incidencia en la deformacion o interferencia del comportamiento de radiofrecuencia, al ser un material no conductor introducido en el campo cercano de los componentes.\\

\begin{figure}
    \centering
    \includegraphics[width=0.8\textwidth]{img/soporte3D5}
    \caption{Union de los doportes de aluminio para el alimentador}
    \label{fig:ensamble4}
\end{figure}

El diseño 3D de la figura \ref{fig:ensamble4} es un soporte que contiene una cabidad centrar cilindrica que conmple la funcion de sostener tanto el alimentador como el receptor por medio de un tubo plastico de PVC\footnote{PVC} que asegura que todo se mantenga alineado con el centro de la parabola, además de permitir un movimiento en el eje de la cabidad cilindrica para ajustar el foco del alimentador.\\

Tiene también 4 ranuras perforadas para aseguirar los soportes con pernos M4 de medida y también 6 perforaciones con cabidades para tuercas M5. Con estas tuercas y con los respectivos tornillos se asegura la poscision del tubo de PVC para fijar el foco una vez encontrado.\\

Las siguientes piezas comparten la misma filosofia de diseño, para poder compatible entre ellas y con el resto de los componentes del telescopio. Además, permiten el rediseño de nuevas piezas para otros alimentadores, cambios de largo en el tubo distriubidor de PVC y en la eleccion de otro material de impreson 3D si se quiciese.\\

\begin{figure}
    \centering
    \includegraphics[width=0.8\textwidth]{img/soporte3D1v1}
    \caption{Interfaz de soporte para el tubo de distribucion y otros elementos}
    \label{fig:ensamble5}
\end{figure}

La figura \ref{fig:ensamble5} es un soporte multiproposito que permite acoplar otros soportes de menor complejidad para ser instalados en la zona del alimentador y receptor. Así permite cambios radicales en la instrumentación que se requiera en el futuro sin tener que rediseñar toda la estructura de sujeción.\\

\begin{figure}
    \centering
    \includegraphics[width=0.8\textwidth]{img/soporte3D7}
    \caption{Soporte interno para electronica de recepción}
    \label{fig:ensamble6}
\end{figure}

El receptor de radiofrecuencia se encuentra dentro de una caja electrica a prueba de agua, pero se requiere un soporte interno para asegurar que la placa de adquisicion de datos y el digitalizador no se muevan y se mantengan en su lugar mientras el telescopio se mueve en distintas elevaciones. La figura \ref{fig:ensamble6} es un soporte que se instala en la caja electrica y permite montar diferentes tipos de receptores y amplificadores.\\

\begin{figure}
    \centering
    \includegraphics[width=0.8\textwidth]{img/soporte3D4}
    \caption{Soporte para la fuente de calibración de la copa de agua \quotes{estrella artificial}}
    \label{fig:ensamble7}
\end{figure}

La figura \ref{fig:ensamble7} es un soporte que fue diseñdo para instalar la fuente de calibración de la copa de agua o "estrella artificial" en parte superior de la copa de agua del cerro Calan. Se divide en 2 piezas que se unen por medio de pernos M4 de plastico para sujetar la antenna circular por presion y con tronillos pasantes. Ademas para poder asegurar este soporte con faciliada y raídez, se diseño la forma de cruz para que por medio de amarras plasticas se pueda asegurar a la baranda de la copa de agua.\\

\begin{figure}
    \centering
    \includegraphics[width=0.8\textwidth]{img/soporte3D2}
    \caption{Soporte para antena circular de alto ancho de banda para cofiguracion de alimentador}
    \label{fig:ensamble8}
\end{figure}

Para las medicones de baja frecuencia (menores a 600 MHz) se debe utilizar la misma antena circular de la figura \ref{fig:ensamble7} pero con un soporte diferente. La figura \ref{fig:ensamble8} es un soporte que permite colocar la antena como alimentador del telescopio por medio del tubo de PVC y asegurarla con pernos M4 al este y pernos plasticos M3 para la antena y el soporte.\\

\begin{figure}
    \centering
    \includegraphics[width=0.8\textwidth]{img/soporte3D6}
    \caption{Soporte para el dipolo exotico como alimentador de 1420MHz}
    \label{fig:ensamble9}
\end{figure}

Al igual que en la figura \ref{fig:ensamble8}, la figura \ref{fig:ensamble9} es un soporte que permite colocar el dipolo exotico como alimentador del telescopio por medio del tubo de PVC. Diferenciandose del sopórte anterior que este diseño permite asegurar la placa de la antena con la deformacion forazada del material impreso, evitando el uso de pernos y tuercas.\\

\subsection{Montura Alt-Azimutal}

El rotor utilizado para la montura alt-azimutal es el modelo \textit{BIG-RAS/HR} de la compañia \textit{RFHamdesign} que esta diseñado para soportar una carga de hasta 319 kg, con una velocidad de movimineto de hasta 2.5 grados por segundo y una resolucion de 0.1 grados para sus encoders.\\

\begin{figure}
    \centering
    \includegraphics[width=0.8\textwidth]{img/soporte_montura}
    \caption{Rotor \textit{BIG-RAS/HR} de la compañia \textit{RFHamdesign} instalada en el pedestal con la montura de acero.}
    \label{fig:ensamble10}
\end{figure}

En la figura \ref{fig:ensamble10} se puede ver el rotor instalado en el pedestal de acero con la montura que hace la interfaz entre los motores y el reflector. La montura tiene unos brazos traseros perforados para instalar los contrapesos de equilibrio y compensar el torque que ejerce la masa del reflector. La pieza que une la montura con el rotor es una tuberia de acero galbanizado de 46.5 mm de diametro, cortada a la medida de la montura. Todos los pernos de sujecion son M10 de cabeza exagonal.\\

\begin{figure}
    \centering
    \includegraphics[width=0.8\textwidth]{img/contrapesos}
    \caption{Montura de acero con los contrapesos de equilibrio instalados.}
    \label{fig:ensamble11}
\end{figure}

En la figura \ref{fig:ensamble11} se pueden ver los contrapesos de equilibrio instalados en la montura de acero. Los contrapesos son ladrillos de plomo y cemento de 10 kg cada uno. Se instalaron 4 Ladrillos en total a una distancia de 78 cm de del eje con la tuberia, lugar donde sin ejercer ninguna fuerza sobre la antena o la montura se equilibra con los pernos de sujecion completamente desajustados.\\

La montura es capaz de moverse en 400 grados en azimut y 180 grados en elevacion, con un rango de movimiento de -40 a 360 en los motores horizontales y de 0 a 180 para los motores verticales. Los cuales para mantener rangos de seguridad la motura se mueve entre 0 a 180 grados en azimuth y elevacion, así se minimiza el reisgo de enrrollado de los cables al girar. En la seccion de software se explicará el funcionamiento del algoritmo de movimiento.\\

\subsection{Rack de control}

El rack de control se ecuentra en el edifico más cercano al telescopio, el edifico del meriano, donde tambien se encuentra el telescopio ARTE. El rack consiste en un gabinete de 12 unidades de rack o \textit{U} completamente de acero tanto su cuerpo como la puerta frontal. Con el objetivo de minimizar el RFI que pueda ser introducido por los componentes electronicos, se utilizo un gabinete completamente cerrado y con una puerta frontal de acero conectado a la tierra local.\\


\begin{figure}
    \centering
    \includegraphics[width=0.8\textwidth]{img/rack}
    \caption{Rack de control con el controlador SPID de la montura y el computador de control.}
    \label{fig:ensamble12}
\end{figure}

En la figura \ref{fig:ensamble12} se puede ver el rack de control con los siguientes elementos ordenados de arriba a abajo: el switch de red, el iyector POE del receptor, la fuente de alimentacion multiple, el controlador de la montura SPID, el computador de control y observacion. Estos elementos se encuentran en la sala de recepcion de ARTE que cuenta con un sistema de climatizacion que mantiene la temperatura a 16 grados celcius constantemente.\\

\section{Alimentador}

Para el alimentador se evaluaron distintas opciones de antenas según su desempeño de ganancia y ancho de banda. Las frecuencias de operacion del telescopio son de 1420 MHz para la banda de hidrógeno y de 300 a 500 MHz para la banda de CHARTS.\\

Como una de las caracteristicas de la cosntruccion del telescopio es la capacidad de intercambiar su alimentador con la estandardizacion de los soportes, se decidio utilizar antenas comerciales que cumplieran con los requerimientos de operacion y se acercaran al rendimiento que declara el fabricante para esta superficie.\\

\subsection{LPDA de alto ancho de banda}

La antena log periodica de dipolo (LPDA) de alto ancho de banda es una antena que se caracteriza por tener una ganancia de 10 dBi y un ancho de banda de 300 a 6000 MHz. Esta antenna se instaló con el elemento más pequeño del arreglo de dipolos en el foco de la parabola. Se orientó verticalmente con respecto al suelo en el telescopio en posicion de azimuth y elevacion de 0 y 0 grados respectivamente.\\

\begin{figure}
    \centering
    \includegraphics[width=0.8\textwidth]{img/lpda}
    \caption{Antena LPDA de alto ancho de banda instalada en el telescopio.}
    \label{fig:ensamble13}
\end{figure}

La antena de la figura \ref{fig:ensamble13} se instaló en el soporte de la figura \ref{fig:ensamble8} y se conectó al receptor por medio de un cable coaxial de 50 ohmios y 1.5 metros de longitud.\\

\subsection{Dipolo exotico}

El dipolo exotico es una antena que forma el arreglo del telescopio ARTE\cite{Gallardo2023} esta antena se caracteriza por tene una ancho de banda de 400 MHz desde 1000 para el diseño impreso instalado en el telescopio. Esta antenna tiene una ganancia de 3dBi y se ubica en el foco de la parabola a 135 cm del la superficie.\\

\begin{figure}
    \centering
    \includegraphics[width=0.8\textwidth]{img/feed}
    \caption{Dipolo de ARTE instalado en el telescopio.}
    \label{fig:ensamble14}
\end{figure}

Esta antena tiene la particularidad de que tiene la ganancia recomendada por el fabricante del reflcetor para utilizar como alimentador a la distancia de 135 cm. Como se puede ver en la figura \ref{fig:ensamble14} la antena de PCB se encuentra instalada en el tubo de PVC con el soporte diseñado de la figura \ref{fig:ensamble9}.\\

\subsection{Antena circular de alto ancho de banda}

Esta antena es la misma que la a utilziar para la fuente de calibracion de la copa de agua, en este caso se quiere utilziar como alimentador en reemplazo de la LPDA de la figura \ref{fig:ensamble13}. Esta antena tiene una ganancia cercana a 3dBi, semejante a la del dipolo de ARTE y lo que se recomienda para el reflector. La antena se instala en el soporte de la figura \ref{fig:ensamble7}.\\

\begin{figure}
    \centering
    \includegraphics[width=0.8\textwidth]{img/paletaFeed}
    \caption{Antena circular de alto ancho de banda instalada en el telescopio.}
    \label{fig:ensamble15}
\end{figure}


\section{Diseño del receptor}

Para el receptor se optó por utilizar una SDR de bajo costo, una filtro pasabanda optimizado para la observacion de la banda de hidrógeno y un amplificador de bajo ruido. Se tomó en cuenta que estos componentes deben no solo ser de alta precision si no que robustos ya que estaran ubicados lo más cerca posible del alimentador a la interperia.\\

\subsection{Cadena de recepción}

Para la cadena de recpecion cuenta con un empaquetado de un amplificador de bajo ruido de la compañia \textit{Noeelec} que contiene ademas un filtro pasabanda de 75 MHz de ancho de banda centrado en 1420 MHz. El amplificador tiene una ganancia tipica a 1.4 GHz de 35 dB con una figura de ruido de 0.6 dB para la misma frecuencia. Ademas este amplificador puede ser alimentado por bias-tee\footnote{} desde la misma SDR\\

\begin{figure}
    \centering
    \includegraphics[width=0.8\textwidth]{img/imagen}
    \caption{Amplificador y filtro pasabanda SAWbird H1 de la compañia \textit{Noeelec}.}
    \label{fig:cadena}
\end{figure}

En la figura \ref{fig:cadena} se puede ver el amplificador y filtro pasabanda SAWbird H1 de la compañia conectado a un analizador de espectro y una fuente de ruido para la caracterizacion de la cadena de recepcion.\\

\subsection{Digitalizador y adquisición}

El digitalizador es una RTL-SDR de la oganizacion \textit{RTL-SDR} basado en el chip R820T de \textit{Rafael Micro} que se conecta por USB a un computador para obtener directamente el voltaje complejo de la IF para que sea procesada por el software de adquisicion. Este digitalizador tiene una frecuencia de muestreo maxima de 3.2 MS/s y una resolucion de 8 bits, pero usualmente se utiliza bajo los 2.56 MS/s para que tenga un comportamiento estable\cite{RTLSDR2018}.\\

\begin{figure}
    \centering
    \includegraphics[width=0.8\textwidth]{img/imagen}
    \caption{RTL-SDR conectada a la cadena de amplificador y una Raspberry PI 4B.}
    \label{fig:digitalizador}
\end{figure}

En la figura \ref{fig:digitalizador} se puede ver la RTL-SDR conectada a la cadena de amplificador y una Raspberry PI 4B con un \textit{Hat} POE. Se utilizo esta configuracion para solo llevar 1 cable ethernet cat 6 por el cual pasaria la energia y los datos. La Raspberry es capaz de alimentar a la SDR que a su vez por medio de su Bias-Tee puede alimentar al amplificador de bajo ruido con la minima cantida posible de cables y conexiones.\\

El cable ethernet utilizado es un cat 6 de 25 metros, que va desde el receptor a el rack de control. Este cable es del tipo FFTP, lo que quiere decir que cada par trensado esta recubierto con una laminade aluminio y a su vez los 4 pares trensados más un conductor de apantallamiento estan recubiertos por otra lamina de alumino que se conecta a tierra en ambos extremos para minimizar el ruido al transportar datos y no producir RFI al telescopio.\\

\section{Software de control y adquisición} \label{sec:software}

La infraestructura digital del telescopio se diseño con un factor principal en mente, que este se pueda operar completamente remoto, por lo que todo el software esta hecho para ser operado desde cualquier lugar con acceso a internet. Accediendo a la terminal de control por medio de SSH\footnote{Secure Shell: } y todos sus sistemas estan conectados a una red local por medio de ethernet.\\

El principal lenguaje de programacion utilizado para el desarrollo de software fue python, por su simplicidad a la hora de generar entornos virtuales de desarrollo y librerias existentes para utilizar los diversos subsistemas, como por ejemplo, el uso de la libreria de \textit{astropy} para los calculos de seguimiento.\\

\subsection{Control de la montura}

El controlador del rotor requiere una comunicacion especifica en hexagesimal para moverse y a travez del mismo protocolo responde con la posicion en la cual se encuentra. Para esto se creó una libreria en python basada en el protocolo Rot2Prog\cite{Rot2Prog} que empaqueta y traduce los comandos de movimiento, elevacion y azimut. Esta libreria es un archivo de python por el nombre de spid.py, la que es importada para todos los demas codigos de control.\\

\paragraph{control.py} es el codigo principal de control, tiene la capacidad de mandar una posicion de azimuth y elevacion, de pedir la posicion actual de la montura, de reiniciar el controlodor en caso que no responda a los comandos y una de las funciones más importantes el parado de emergencia de cualquier movimiento.\\

\paragraph{cpt\_traking\_software.py} Este es un software más sofisticado que se creo para el seguimiento de cuerpos celestes. a partir de las coordenadas de declinacion y ascencion recta, el software calcula la posicion de la montura para este astro segun la ubicacion del telescopio y la hora local.\\

\begin{figure}
    \centering
    \includegraphics[width=0.8\textwidth]{img/traking}
    \caption{cpt\_traking\_software.py en funcionamiento.}
    \label{fig:control}
\end{figure}

En la figura \ref{fig:control} se puede ver el software en funcionamiento, con las opciones de cambiar el objeto a seguir \textit{change}, el seguimiento del objeto \textit{follow} y el parado del movimiento\textit{stop}. El software es capaz de invertir la posicion de elevacion del telescopio para minimizar el movimiento en azimut y evitar que los cables puedan enrollarse entre si.\\

Por ejemplo si al calcular que la posicion del astro en elevacion de 50 grados y azimuth de 315, requiere un movimiento de más de 180 grados en azimut, el software invierte la posicion de elevacion para que el movimiento sea menor a 180 grados resultando en que el telescopio apunte a 130 grados de elevacion y a 135 grados en azimut. De forma automatica, si el astro tiene una elevacion menor a los 30 grados, o mayor a 150 grados en la inversion, este deja de seguir el astro ya que a esta elevacion la interferencia de radio es muy notoria en las observaciones por efectos atmosfericos y cada vez entra en la linea de vista elementos de comunicaciones inhalambricas terrestres.\\

\subsection{Adquisición de datos} 

Para la adquisicion de datos se crearon 2 softwares en python para esta tarea, una para adquirir una acumulacion de espectros para las observaciones de un objeto celeste y otro para la calibracion del instrumento.\\

\paragraph{rtl\_spectra.py} Es un script que utiliza la radio RTL-SDR para obtener espectros mediante un comando de especifico, este se utiliza principalmente para las caracterizaciones y las mediciones. Este se utiliza en conjunto con el software de control para crear las variantes de medicion de patro de radiacion. Se puede configurar la taza de datos, el tamaño de la FFT, la cantidad de espectros tomados y el formato de guardado.\\

\paragraph{cpt\_rtl\_adquisition.py} Es el software de observacion, el cual tiene un \textit{Ring buffer} o un acumulador de espectros flotantes, esto quiere decir que, segun como se configure, puede acumular una cantidad de espectros que se van actualizand constantemente con nuevos y eliminando los viejos en la ventana de tiempo que se requiera o en lo que la memoria pueda guardar.\\

Al igual que el script anterior este se puede configurar para la taza de datos, el tamaño de la FFT, la cantidad de espectros tomados y el formato de guardado. Este, por otra parte, esta diseñado para obtener una gran cantidad de espectros para ser guardados en una estructura eficiente en espacio en codigo binario. Tambien tiene la tarea de mostrar en tiempo real la acumulacion promediada de los espectros y en paralelo obtener las muestras y agregarlas al ring-buffer.\\


\section{Infraestructura de caracterizacion}

Para caracterizar el telescopio se requieren de una serie de instrumentos y elementos que permitan obtener los datos necesarios para la calibracion y la verificacion de los resultados obtenidos.\\

\subsection{Fuente de calibración}

Ya que el campo lejano de las antenas electricamente grandes, como es el caso de una antena de apertura. Para medir el patron de radiacion en potencia de una antena se requiere de una fuente de radiofrecuencia conocida a una distancia mayor a la de campo lejao de la antena que se quiere medir.\\

Para esto se instaló una antena de alto ancho de banda (192 MHz a 8 GHz) en la copa de agua del cerro calan, con un cable coaxial de 20 metros de longitud. Puediendo así dejar la anetna instalada en la parte superior y poder conectar generadores de señales desde la parte inferior. Esta antena le llamaremos la \quotes{estrella artificial}.\\

\begin{figure}[h!]
    \centering
    \begin{subfigure}{0.45\textwidth}
        \includegraphics[width=\textwidth]{img/paleta}
        \caption{Antena de polarizacion circular de alto ancho de banda con su soporte para la copa de agua.}
        \label{fig:antena_estrella}
    \end{subfigure}
    \begin{subfigure}{0.45\textwidth}
        \includegraphics[width=\textwidth]{img/fake_star}
        \caption{Antena de la estrella artificial instalada en la copa de agua.}
        \label{fig:antena_estrella2}
    \end{subfigure}
\end{figure}

La antena de la estrella artificial se encuentra a una altura de 15 metros sobre el suelo y a 186 metros de la antena del telescopio. La antena de la estrella artificial es una antena de polarizacion circular de alto ancho de banda con una ganancia de 3 dBi aproximadamente.\\

La línea de vista de la antena se encuentra totalmente despejada, manteniendo la primera zona de Fresnel libre de obstaculos para las frecuencias de interes.\\

Como generador se señales se utilizo un generador Valon 5008 con una salida de 2.23 dBm a 1428 MHz y a 400MHz. Ademas se le isntalo un filtro pasabajo para minimizar la presencia de los armonicos de alta frecuencia evitando la generacion innecesaria de RFI.\\

\begin{figure}
    \centering
    \includegraphics[width=0.8\textwidth]{img/valon}
    \caption{Generador de señales Valon 5008 con filtro pasabajo con una bateria externa.}
    \label{fig:generador}
\end{figure}

El generador de la figura \ref{fig:generador} se conecta a la antena de la estrella artificial por medio de un cable coaxial de 20 metros de longitud y se alimenta por una bateria externa de 5 V1. Se programa previamente la frecuencia a la que se requiera para las mediciones.\\

\subsection{Fuente de ruido}

Para realizar medicion de la temperatura de ruido se requiere de una fuente de ruido. Para obtener la temperatura de la cadena de recepcion se utilizo una fuente de ruido Agilent 346B con una aliemntacion de 28 V.\\

\begin{figure}
    \centering
    \includegraphics[width=0.8\textwidth]{img/fuenteRuido}
    \caption{Fuente de ruido Agilent 346B.}
    \label{fig:fuente_ruido}
\end{figure}

\subsection{Software de caracterización}

\paragraph{cpt\_rp\_measure.py} Es un software que al igual que los de la seccion \ref{sec:software} obtiene espectros y los guarda para el analisis futuro. La diferencia es que este software esta diseñado para la calibracion del instrumento, por lo que ademas este instrumento guardas los espectros tomados por angulo con respecto a la estrella artificial de la copa de agua para las mediciones de patron de radiacion.\\

Este script genera un archivo con los espectros tomados por angulo y luego mueve la montura a otro angulo para tomar otro espectro, este proceso se repite hasta que se obtienen un corte de 180 grados con la cantidad de espectros que haya sido configurada.\\

Como la fuente de calibracion se encuentra en altura, hay que ajustar el plano de rotacion con respecto a plano azmutal de la montura, para esto se utiliza la siguiente conversion de coordenadas:\\

\begin{equation}
    \theta' = \theta + \left(1- \frac{2\phi}{\pi}\right)E
\end{equation}

\begin{equation}
    \phi' = \phi
\end{equation}

Donde $\theta$ es la elevacion original, $E$ es el angulo de elevacion de la estrella artificial con respecto a la antena, $\phi$ es el azimut, $\theta'$ y $\phi'$ son las nuevas coordenadas. Con esta conversion se tiene una elevacion específica para cada punto de azimut que permite mantener el plano de rotacion de la fuente de calibracion en el nuevo plano de azimutal.\\

\paragraph{cpt\_siglent.py} ES un software que utiliza de manera remota el instrumento Siglent SVA1075X para obtener sus espectros y realizar las mismas mediciones de patron de radiacion que el software anterior.\\
enfoque1\chapter{Caracterización y puesta en servicio}

En este capitulo se expondran los procesos de caracterizacion y puesta en servicio del telescopio CPT. Para abordar los puntos de caracterizacion y primera luz de los objetivos propuestos para este trabajo. Se detallaran los aspectos considerados para cada una de las mediciones y los fundamentos correspondientes.

\section{Enfoque del alimentador}

Se relizaron 2 mediciones de enfoque del alimentador, una a 70 metros y otra a 186 metros. Para la primera medicion se utilizo un generador de senales generico con una LPDA de bajo ancho de banda. Luego para el resto de las mediciones se utilizo la estrella artificial de la copa de agua de la seccion XX.\\

\subsection{Alimentador sin soportes a 70 metros}

En principios el alimentador de telesecopio consistia en una antena LPDA (Log Periodic Dipole Array) de 296 MHz a 6 GHz, de ultra ancho de banda, con una ganancia de aproximadamente 9dBi. Con el receptor instalado en esta antena, se procedio a realizar el enfoque del alimentador. Para esta etapa se retiraro el soporte tetrapodo y se instalo el alimentador en un tripode auxiliar sostenido por un tubo de PVC para lograr la altura del centro del reflector de 2 metros.\\

\begin{figure}
    \centering
    \includegraphics[width=0.7\textwidth]{img/enfoque1}
    \caption{Antena LPDA en tripode auxiliar a 2 metros de altura.}
    \label{fig:antena_lpda}
\end{figure}

Con la antena de la figura \ref{fig:antena_lpda} se procedio a realizar el enfoque del alimentador. La medicion consiste en mover el alimentador en el eje Z, es decir en la direccion de la apertura del reflector. A una distancia de 70 metros desde el relfector se instaló sobre otro tripode un generador de señales portatil con otra LPDA de menor ancho de banda.\\

\begin{figure}[h!]
    \centering
    \begin{subfigure}{0.45\textwidth}
        \includegraphics[width=\textwidth]{img/enfoque_cerca}
        \caption{Generador de señales portatil con la antena orientada hacia el reflector a 70 metros.}
        \label{fig:generador}
    \end{subfigure}
    \begin{subfigure}{0.45\textwidth}
        \includegraphics[width=\textwidth]{img/enfoque_cerca1}
        \caption{Antena LPDA de menor ancho de banda instalada con el generador de señales en tripode.}
        \label{fig:antena_lpda}
    \end{subfigure}
\end{figure}

El generador de señales se configuro a una frecuencia de 1000MHz y se disparo constantemente el tono en dicha frecuencia. Se instalo a una distancia de 70 metros dentro del praque cerro Calan con la consideracion que por la apertura de 3 metros del reflector se obtiene un campo lejano de 60.5 metros a 1000MHz, por lo que este generador se encontraba en el campo lejano del reflector.\\

\begin{figure}
    \centering
    \includegraphics[width=0.7\textwidth]{img/70m_measure}
    \caption{Distancia de 70 metros entre el reflector y el generador de señales con elevaciones similares.}
    \label{fig:70m_measure}
\end{figure} 

En la figura \ref{fig:70m_measure} se muestra la distancia aproximada de 70 metros, tomando 10 metros de distancia adicional para asegurar el campo lejano a dicha frecuencia. Se alineo visualmente la antena del generador con el reflector a la distancia, la que se encuentra parcialemente con una linea de vista obstaculisada por arboles y arbustos.\\

Para efectos de la medicion, como se quiere encontrar un punto aproximado de enfoque la exactitud de esta medicion no es critica. Se procedio a mover el alimentador en el eje Z, es decir en la direccion de la apertura del reflector, hasta encontrar el punto donde la señal del generador era maxima.\\

La potencia recibida por el receptor fue medida con el software de medicon utilizando la RTL-SDR guardando los espectros para distancias de 5 cm en 5 cm. Se midio la razon señal a ruido y se obtuvo el punto de maxima potencia recibida en dBFS. Medidas que luego son calibradas en dBm por la medicion de sensibilidad.\\

\subsection{Alimentador con soportes con la estrella artificial}

Para la medicion de enfoque con la estrella artificial se instalo el soporte tetrapodo y se coloco el alimentador en su posicion final despues de las mediciones anteriores. Se configuro generador de señales valon de la estrella artificial a 1428MHz. Se utiliza esta nueva frecuencia para evitar interferencia de radio frecuencia en la medicion, utilziando el filtro delgado de H1.\\

Para el caso de 1428MHz, la distancia de campo lejano es de 85 metros y la estrella artificial se encuentra a 186 metros, estando perfectamente en el campo lejano del reflector. Ademas la alineacion con el reflector se realzia con el control automatico de la montura alt azimutal y la liena de vista se encuentra completamente libre para la priemra zona de fresnel.\\

\begin{figure}
    \centering
    \includegraphics[width=0.7\textwidth]{img/feed_focus}
    \caption{Alimentador dipolo exotico con tetrapodo isntalado.}
    \label{fig:enfoque2}
\end{figure}

En la figura \ref{fig:enfoque2} se muestra el alimentador con el tetrapodo instalado y un tubo PVC milimetrado para medir la distancia relativa al reflector, usando de referencia el punto donde los soportes se unen al tubo del alimentador.\\

Para esta medicion de enfoque, se movio el alimentador en el eje Z con una resolucion de 0.5 cm hasta encontrar el punto de maxima potencia recibida. Se midio la razon señal a ruido y se obtuvo el punto de maxima potencia recibida en dBFS al igual que en la medicion anterior.\\

\section{Medicion del patrón de radiación}

Se utilizaron 2 plataformas para la medicion del patron de radiacion, una con la medida relativa dBFS obtenida de los espectros de la RTL-SDR y otra con la medida absoluta en dBm obtenida con el analizador de espectros Siglent SVA1075x. Para todas las mediciones se utilizó la estrella artificial de la copa de agua.\\

Todas las mediciones de patron de radiacion se realizaron en la cima del cerro Calan, en la plataforma de observacion del telescopio CPT. Para cada frecuencia medida, se hicieron los cortes azimutales y de elevecaion, o la medida del campo H y el campo E. Todos los cortes son de 180 grados para obtener con claridad los lobulos laterales y con una definicion de 1 punto por grado.\\


\subsection{Medicion relativa para banda de H1}

La medicion con la RTL-SDR se realizo a 1428MHz, utilizando el filtro angosto de la misma frecuencia de radioastronomy suplies. El generador Valon, se configuro a 1428MHz con una potencia inyectada a la estrella artificial de 0.23dBm.\\

Las perdidas ohmnicas del cable coaxial RG316 a 1428MHz son de 10.5 dB por 10 m, como el cable que alimenta la antena de la copa de agua es de 20 metros, se obtiene una perdida de 21 dB. La potencia recibida por la antena es de -21 dBm aproximadamente\\

Con los software de medicion se obtuvieron los espectros de la señal recibida por la RTL-SDR, se midio la razon señal a ruido y se guardaron 1000 espectros para cada grado de elevacion y azimut. Como la copa de agua se encuentra en altura, se genera un plano semicircular elevado en 7 grados sobre el eje horizontal, donde se corrigen los valores de elevacion por angulo azimutal con la ecuacion de correccion \ref{eq:powerdensity}. Para el segundo corte, se genera un ofset de 7 grados en elevacion, y para medir de 0 a -90 grados, se invierte la posicion azimutal en 180 grados para obtener ese cuadrante.\\

\subsection{Medicion absoluta para todas las badnas de interes}

\section{Sensibilidad y temperatura de ruido}

La medida de sensibilidad se realizó inyectando un tono para cada frecuencia de interes en la entrada de la cadena de recepcion. Se utilizó el generador de señales Rode and Schwartz SMB100A con una potencia de salida de -80 dBm y un coaxial RG316 de 10 metros al receptor instalado en el foco de la antena. Se midieron los espectros generados por la RTL-SDR para determinar la escala dBFS de la radio y calibrar los demás espectros en potencia.\\

Se guardaron los espectros de 300MHz, 400MHz y 500MHz para cubrir la banda de interes del proyecto CHARTS. Tambien se guardaron los espectros de 1000MHz, 1428MHz, 1500MHz y el límite de digitalizacion de 1700MHz.\\

\subsection{Medicion de la temperatura de ruido}

Para medir la temperatura de ruido y por ende la figura del receptor, se utilizó la fuente de ruido Agilent 346B con una amplificacion de 40dB. Para la cadena de amplificacion se utilizó el LNA + Filtro H1 SAWbird+ H1 de Nooelec, el cual se conectó a la fuente de ruido y se midio la potencia de ruido en la salida en el analizador de espectro.\\

Para la temperatura de ruido del receptor, se inyectó la señal de ruido en la entrada del receptor y se midio la potencia de ruido con los espectros de la RTL-SDR en la banda de interes.\\

Para obtener la temperatura del sistema completo se realizó una acumulacion de espectros del centro de la galaxia a 1420MHz y a una region del cielo limpia en concentracion de hidrógeno neutro. Consultando a los catalogos de radioastronomia se obtuvo la temperatura de la galaxia y de la region del cielo.\\

Con cada una de estas mediciones se realizó el cálculo de temperatura de ruido con el metodo de Y-factor y se obtuvo la figura de ruido de la cadena de amplificacion, del receptor y del sistema.\\


\section{Medicion del error de apuntamiento}

\section{Primera luz}

Para la primera luz, se escojio el centro de la galaxia por varias razones. La primera es que es una fuente de radio muy fuerte, facil de detectar y es bastamente estudiada y catalogada. La segunda razon es que para la epoca de medicion, noviembre, diciembre y enero, el centro de la galaxia se encuentra bastante cerca del cenit para la latitud de Santiago, lo que facilita la observacion.\\

El centro de la galaxia en una asencion recta de 17 h 45 m 40 s y una declinacion -29°00'28'', coordinendas que son ingresadas en el software de control de la montura alt azimutal. Se acumularon espectros por periodos de 2 horas por 3 dias, desde las 10 am hasta las 6 pm, para obtener la maxima cantidad de datos posibles. Tambien se descartaron algunas observaciones deonde el solo se encontraba muy cerca del centro de la galaxia, para evitar la saturacion del receptor.\\

Para la observacion se configuro el telescopio con el receptor de H1 y con la antena dipolo exotico en el foco geometrico del reflector. Se obtuvieron los espectros de la RTL-SDR y se guardaron en el disco duro para su posterior analisis.\\
\chapter{Analisis de Resultados}

\section{Posicion del alimentador}

\section{Patrón de radiación}

\section{Ganancia y Directividad}

\section{Sensibilidad}

\section{Error de apuntamiento}

\section{Ancho de banda}

\section{Primera luz}
\chapter{Conclusiones}

\section{Trabajos Futuros}

\bibliography{library}

\begin{appendixs}

\section{Python scripts}

\subsection{cpt\_tracking\_software.py}

Este codigo permite el seguimiento de un objeto en el cielo, utilizando coordenadas altazimutales. El programa se ejecuta en un hilo separado y actualiza la posición del telescopio cada 5 segundos. El usuario puede ingresar comandos para iniciar o detener el seguimiento, cambiar el objetivo o salir del programa.

\begin{sourcecode}[]{python}{}
    import time
    import threading
    from astropy.coordinates import EarthLocation, AltAz, SkyCoord
    from astropy.time import Time
    import spid
    import socket
    from astropy import units as u
    
    # Define your telescope's location
    latitude = -33.395720  # Replace with your latitude in degrees
    longitude = -70.536856  # Replace with your longitude in degrees
    elevation = 868  # Replace with your elevation in meters
    
    # Telescope limits (adjust based on your telescope's safety and design)
    ALT_MIN = 20  # Minimum Altitude in degrees (avoid negative or below horizon)
    ALT_MAX = 150  # Maximum Altitude in degrees
    AZ_MIN = 0   # Minimum Azimuth in degrees
    AZ_MAX = 261 # Maximum Azimuth in degrees
    
    # Global variables for tracking and target
    tracking = False
    current_target = None
    stop_threads = False
    
    def send_angle(alt, az, host='10.17.89.223', port = 23):
        with socket.socket(socket.AF_INET, socket.SOCK_STREAM) as client_socket:
            client_socket.connect((host, port))
            message = spid.encode_command(spid.build_command(round(float(az),1), round(float(alt))))
            client_socket.sendall(message)
    
    
    def get_altz_coords(ra, dec):
        radec = SkyCoord(ra=ra*u.hour, dec=dec*u.deg, frame='icrs')
        # Calan GEOGRAPHIC_COORD
        calan_obs = EarthLocation(lat=-33.3961*u.deg, lon=-70.537*u.deg, height=867*u.m)
        now = Time.now()
        altaz_frame = AltAz(obstime=now,location=calan_obs)
        altaz = radec.transform_to(altaz_frame)
        alt = altaz.alt.deg
        az = altaz.az.deg
    
        if az>200:
            az = 360 -az
            alt = 180 -alt
        return alt, az
    
    
    def is_within_limits(alt, az):
        """Check if the Alt/Az coordinates are within the telescope's limits."""
        return ALT_MIN <= alt <= ALT_MAX and AZ_MIN <= az <= AZ_MAX
    
    
    def track_target():
        """Background thread function to update tracking and send data via socket."""
        global stop_threads
    
        while not stop_threads:
            if tracking and current_target:
                ra, dec = current_target["ra"], current_target["dec"]
                alt, az = get_altz_coords(ra, dec)
    
                if is_within_limits(alt, az):
                    print(f"Tracking {current_target['name']}: Altitude = {alt:.2f}°, Azimuth = {az:.2f}°")
                    data = f"{current_target['name']}: Altitude = {alt:.2f}, Azimuth = {az:.2f}\n"
                    try:
                        send_angle(alt, az)
                    except:
                        print("The controller is not responding")
                else:
                    print(f"{current_target['name']} is outside the telescope limits: Alt = {alt:.2f}°, Az = {az:.2f}°")
            time.sleep(5)  # Update every 5 seconds
    
    
    
    def main():
        global tracking, current_target, stop_threads
    
        print("Welcome to the telescope tracking program!")
        print("Commands: follow, stop, change, exit")
    
        # Start the tracking thread
        tracking_thread = threading.Thread(target=track_target, daemon=True)
        tracking_thread.start()
    
        while True:
            command = input("\nEnter command (follow, stop, change, exit): ").strip().lower()
    
            if command == "follow":
                if current_target:
                    tracking = True
                    print(f"Started following {current_target['name']}...")
                else:
                    print("No target selected. Use 'change' to select a target.")
    
            elif command == "stop":
                tracking = False
                print("Tracking stopped.")
    
            elif command == "change":
                name = input("Enter the name of the object: ").strip()
                ra = float(input("Enter the Right Ascension (RA) in degrees: "))
                dec = float(input("Enter the Declination (Dec) in degrees: "))
    
                current_target = {"name": name, "ra": ra, "dec": dec}
                print(f"Target changed to {name} (RA: {ra}, Dec: {dec}).")
    
            elif command == "exit":
                print("Exiting the program. Goodbye!")
                stop_threads = True
                tracking_thread.join()
                socket_thread.join()
                break
    
            else:
                print("Invalid command. Please try again.")
    
    
    if __name__ == "__main__":
        main()
    
\end{sourcecode}

\subsection{spid.py}

Este script funciona como libreria para codificar y decodificar comandos para el protocolo SPID MD-1 Rot2Prog. Permite convertir ángulos a pulsos, construir comandos de movimiento al stream del socket UDP y decodificar respuestas del controlador.

\begin{sourcecode}[]{python}{}
    # Library to process all bytes formating for the SPID MD-1 Rot2Prog protocol

    deg_res = 2 # 01 - 1, 02 - 0.5, 04 - 0.25, byte - degrees per pulse
    stop_str = "57000000000000000000000F20"
    status_str = "57000000000000000000001F20"
    command_str = "5730303030PH30303030PV2F20"
    restart_str = "57EFBEADDE000000000000EE20"
    
    def angle_to_pulse(angle):
        i = str(int(deg_res * (360 + angle))) # create string of pulses
        if len(i) < 4: # Add the "0" for complet the byte string
            i = "0" + i
        return i
    
    def build_command(az, el):
        command = list(command_str)
        az_pulse = angle_to_pulse(az)
        el_pulse = angle_to_pulse(el)
    
        command[3] = az_pulse[0]
        command[5] = az_pulse[1]
        command[7] = az_pulse[2]
        command[9] = az_pulse[3]
    
        command[10] = "0"
        command[11] = str(deg_res)
    
        command[13] = el_pulse[0]
        command[15] = el_pulse[1]
        command[17] = el_pulse[2]
        command[19] = el_pulse[3]
    
        command[20] = "0"
        command[21] = str(deg_res)
        
        return "".join(command)
    
    def encode_command(msg):
        return bytes.fromhex(msg)
    
    def decode_command(msg):
        response_string = msg.hex()
        H1 = int(response_string[3:4], 16)
        H2 = int(response_string[5:6], 16)
        H3 = int(response_string[7:8], 16)
        H4 = int(response_string[9:10], 16)
        V1 = int(response_string[13:14], 16)
        V2 = int(response_string[15:16], 16)
        V3 = int(response_string[17:18], 16)
        V4 = int(response_string[19:20], 16)
    
        # Calculate angles for Az/El
        az = round((H1 * 100) + (H2 * 10) + H3 + (H4 / 10) -360, 1)
        el = round((V1 * 100) + (V2 * 10) + V3 + (V4 / 10) -360, 1)
    
        return (az, el)    
\end{sourcecode}

\subsection{control.py}

Este script permite la comunicación con el controlador del telescopio a través de un socket UDP. Permite enviar comandos para mover el telescopio, detenerlo y consultar su estad. A su vez tiene los parametros de movimiento que aseguran que cada movimiento no ponga en riesgo los cables con el pedestal. También incluye un hilo para recibir mensajes del controlador y otro para enviar los comandos.

\begin{sourcecode}[]{python}{}
import spid
import socket
import numpy as np
import threading
import sys
import time

movement_monitor = False
stop_threads = False
position = 0,0
stop_position = 0,0

# Function to handle receiving messages from the server
def receive_messages(sock):
    global stop_threads
    global position
    global stop_position

    sock.settimeout(1.0)
    while not stop_threads:
        if stop_threads:
            break
        try:
            message = sock.recv(1024)
            if message:
                position = spid.decode_command(message)
                print(f"Current position -> EL: {position[1]} AZ: {position[0]}")

            else:
                print("Connection closed by the server.")
                stop_threads = True
        
        except socket.timeout:
            continue

        except OSError:
            break

        except ConnectionAbortedError:
            print("Connection was closed.")
            break
        except Exception as e:
            print(f"Error: {str(e)}")
            break

# Function to handle sending messages to the server
def send_messages(sock):
    global stop_threads
    global position
    global stop_position
    global movement_monitor

    while not stop_threads:
        message = input("").split(" ")
        
        if movement_monitor:
            msg = spid.encode_command(spid.status_str)
            if stop_position == position:
                movement_monitor = False

        if message[0] == "stop":
            msg = spid.encode_command(spid.stop_str)
            print("Stoping Telescope Movement")

        if message[0] == "status":
            msg = spid.encode_command(spid.status_str)
            print(f"Actual Position: ")

        if message[0] == "park":
            msg = spid.encode_command(spid.build_command(0,90))
            print("Moving to park position")

        if message[0] == "service":
            msg = spid.encode_command(spid.build_command(0,0))
            print("Moving to service position")

        if message[0] == "restart":
            msg = spid.encode_command(spid.restart_str)

        if message[0] == "move":
            if len(message) < 2:
                print("The command has missing arguments: move [move type] [position] ... ")
            elif message[1] == "elaz":
                if len(message) < 4:
                    print("Elevation and Azimuth movement needs the 2 position arguments")
                else:
                    msg = spid.encode_command(spid.build_command(str_to_f(message[3]),str_to_f(message[2])))
                    stop_pos = message[2], message[3]
                    print(f"Moving to El - Az {message[2]} - {message[3]}")
                    movement_monitor = True

            elif message[1] == "el":
                if len(message) < 3:
                    print("Elevation movement missing argument")
                else:
                    msg = spid.encode_command(spid.build_command(str_to_f(position[0]), str_to_f(message[2])))
                    print(f"Moving to El {message[2]}")
            elif message[1] == "az":
                if len(message) < 3:
                    print("Azimuth movement missing argument")
                else:
                    msg = spid.encode_command(spid.build_command(str_to_f(message[2]), str_to_f(position[1])))
                    print("Moving to Az {message[2]}")

        if message[0].lower() == 'exit':
            print("Closing connection...")
            stop_threads = True
            sock.close()
            break

        try:
            sock.sendall(msg)
        except Exception as e:
            print(f"Error: {str(e)}")
            stop_threads = True
            break
    sys.exit()

def str_to_f(x):
    return round(float(x),1)

def follow_ra_dec(ra,dec):
    pass


def main():
    global stop_threads
    global position
    global stop_position

    # Define server address and port
    host = "10.17.89.223"  # or '127.0.0.1' or server IP address
    port = 23        # Ensure this port matches the server's port

    # Create a socket object
    sock = socket.socket(socket.AF_INET, socket.SOCK_STREAM)

    try:
        # Connect to the server
        sock.connect((host, port))
        print(f"Connected to server {host}:{port}")

        # Read the actual position of the antenna
        sock.sendall(spid.encode_command(spid.status_str))
        time.sleep(0.01)
        angles = spid.decode_command(sock.recv(1024))
        position = angles
        #angles = position
        print(f"Actual Position: Elevation: {position[1]} - Azimuth: {position[0]}")
    
    except Exception as e:
        print(f"Connection error: {str(e)}")
        return

    # Start the receiving thread
    receive_thread = threading.Thread(target=receive_messages, args=(sock,))
    receive_thread.start()

    # Start the sending thread (this is the main thread)
    send_messages(sock)

    # Wait for the receiving thread to finish (if connection closed)
    receive_thread.join()

    sys.exit()

if __name__ == "__main__":
    main()


\end{sourcecode}

\subsection{astrortl.py}

Este codigo contiene funciones para conectarse a un receptor RTL-SDR, configurar sus parámetros y recibir datos. También incluye funciones para procesar la serie muestras de IF y calcular el espectro de frecuencias.

\begin{sourcecode}[]{python}{}
    import socket
    import time
    import numpy as np
    import struct
    import time
    
    def send_command(socket, command, value):
        socket.send(struct.pack('>BI', command, int(value)))
        
    def set_up_rtl(socket, central_freq, sample_rate= 2.048e6, gain_mode=0, gain_index=29, freq_correction = 0):
        #Set up central freq
        send_command(socket, 0x01, central_freq)
        print(f"Central freq set to: {central_freq}")
    
        #Set sample rate
        send_command(socket, 0x02, sample_rate)
        print(f"Sample Rate set to: {sample_rate}")
    
        #Set gain mode
        send_command(socket, 0x03, gain_mode)
        print(f"Gain mode set to: {gain_mode}")
    
        #Set gain by index
        send_command(socket, 0x0d, gain_index)
        print(f"Gain index set to: {gain_index}")
    
        #Set freq correction
        send_command(socket, 0x05, freq_correction)
        print(f"Freq correction set to: {freq_correction}")
    
    def connect_to_rtl(host="127.0.0.1", port=1234):
        sdr_socket = socket.socket(socket.AF_INET, socket.SOCK_STREAM)
        sdr_socket.connect((host, port))
        return sdr_socket
    
    def receive_data(socket, samples=4096):
        buffer_size = samples
        socket.recv(2048)
        raw_data = b''
        
        while len(raw_data) < buffer_size:
            chunk = socket.recv(buffer_size - len(raw_data))
            if not chunk:
                raise RuntimeError("Socket connection closed while receiving data.")
            raw_data += chunk
        #data = np.ctypeslib.as_array(raw_data)
        #iq = raw_data.astype(np.float64).view(np.complex128)
        #iq /= 127.5
        #iq -= (1 + ij)
    
        iq  = [complex(i/(255/2) - 1, q/(255/2) - 1) for i, q in zip(raw_data[::2], raw_data[1::2])]
    
        return iq
    
    def compute_fft(iq_samples, sample_rate):
        fft_size = len(iq_samples)
        spectrum = np.fft.fftshift(np.fft.fft(iq_samples, fft_size))
        spectrum_power = 20 * np.log10(np.abs(spectrum) + 1e-12)  # Convert to dB with numerical stability
        freqs = np.fft.fftshift(np.fft.fftfreq(fft_size, d=1/sample_rate))
        return freqs, spectrum_power
\end{sourcecode}

\subsection{cpt\_rtl\_adquisition.py}

Este script automatiza la observación configurando el receptor y definiendo el tamaño de cada espectro obtenido y a su vez controla el guardado de los espectros en un archivo .npy. El script se ejecuta indefinidamente, guardando los espectros cada 10000 muestras y reiniciando el buffer de datos o hasta que el usuario termine la observación.

\begin{sourcecode}[]{python}{}
import astrortl as rtl
import numpy as np
import matplotlib.pyplot as plt
from collections import deque
import threading
import time

fft_size = 4096
samples = 0
sdr_socket = rtl.connect_to_rtl(host="10.17.89.224", port=1234)
rtl.set_up_rtl(sdr_socket, 1420e6)
file = 0

buffer = []

def data_acquisition(socket, buffer, fft_size):
    freqs, fft = rtl.compute_fft(rtl.receive_data(socket, fft_size), fft_size)
    fft = fft - 80
    buffer = np.append(buffer, fft)
        #time.sleep(0.1)  # Simulate a 10 Hz sampling rate

while True:
    try:
        freqs, fft = rtl.compute_fft(rtl.receive_data(sdr_socket, fft_size), fft_size)
        fft = fft - 80
        buffer.append(fft)
        if samples == 10000-1:
            np.save(f"spectras/milky_way_{file}.npy", np.array(buffer))
            buffer = []
            print(f"Saving buffer {file}")
            file = file + 1

            samples = 0
        else:
            samples = samples + 1
    except:
        sdr_socket.close()
\end{sourcecode}

\subsection{spectra\_avg.py}

\begin{sourcecode}[]{python}{}
    import numpy as np
    import os

    avg_spectra = []
    file = 1
    
    while True:
        files = os.listdir('spectras')
        files = [s for s in files if ".npy" in s]
        try:
            if len(files)>4:
                current = f'spectras/{min(files)}'
                data = np.load(current)
                mean = np.mean(data, axis=0)
                avg_spectra.append(list(mean))
                os.remove(current)
                print(len(avg_spectra))
        except: pass
            
        if len(avg_spectra) == 100:
            np.save(f"spectras/sun_avg_{file}.npy", np.array(avg_spectra))
            avg_spectra = []
            print(f"Saving avg buffer {file}")
            file = file + 1
    
            samples = 0
            
        if len(os.listdir('spectras'))>10:
            print('Too many files')
            break
            

\end{sourcecode}

\subsection{rp\_rtl\_spectrum.py}

Script utilizado para medir el patrón de radiación con la fuente de calibración, incluyendo todas las conversiones de elevación y azimuth para mantener la referencia del patrón



\begin{sourcecode}[]{python}{}
import socket
import time
import numpy as np
import struct
import matplotlib.pyplot as plt
import time
import spid
import decimal
import csv
import pandas as pd
import astrortl

def angle_azimuth_conv(az):
    return 261 - (90 - az)

def angle_elevation_conv(az):
    return (1-2*(abs(az)*np.pi/180)/np.pi)*7

def angle_azimuth_conv_el(el):
    if el < 0:
        return 180 + 171

    else:
        return 171

def angle_elevation_conv_el(el):
    if el < 0:
        
        return 173 + el

    else:
        return el + 7
        

def send_angle(az, host='10.17.89.223', port = 23):
    with socket.socket(socket.AF_INET, socket.SOCK_STREAM) as client_socket:
        client_socket.connect((host, port))
        message = spid.encode_command(spid.build_command(round(float(angle_azimuth_conv(az)),1), round(float(angle_elevation_conv(az)))))
        client_socket.sendall(message)

def send_angle_el(el, host='10.17.89.223', port = 23):
    with socket.socket(socket.AF_INET, socket.SOCK_STREAM) as client_socket:
        client_socket.connect((host, port))
        message = spid.encode_command(spid.build_command(round(float(angle_azimuth_conv_el(el)),1), round(float(angle_elevation_conv_el(el)))))
        client_socket.sendall(message)

\end{sourcecode}

\newpage

\section{Diagramas mecanicos}

\includehfpdf[scale = 0.7 ,pages=-,pagecommand={\subsection{Soporte para tetrápodo}\label{fig:soporte_patas}},linktodoc=true]{draws/soporte_patas}
\includehfpdf[scale = 0.7 ,pages=-,pagecommand={\subsection{Soporte universal para el alimentador}\label{fig:soporte_feed}}]{draws/soporte_feed}
\includehfpdf[scale = 0.7 ,pages=-,pagecommand={\subsection{Soporte para estrella artifical}\label{fig:paleta}}]{draws/paleta}
\includehfpdf[scale = 0.7 ,pages=-,pagecommand={\subsection{Soporte para alimentador de alto ancho de banda}\label{fig:paleta_feed}}]{draws/paleta_feed}
\includehfpdf[scale = 0.7 ,pages=-,pagecommand={\subsection{Soporte para alimentador del dipolo de ARTE}\label{fig:soporte_patch}}]{draws/soporte_patch} 

\end{appendixs}


% FIN DEL DOCUMENTO
\end{document}