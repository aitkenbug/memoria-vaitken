\chapter{Conclusiones}

Se comisionó un nuevo radiotelescopio con tecnologías accesibles y de bajo costo, como las radios definidas por software (SDR). Se pudo desarrollar un sistema de 16348 canales espectrales a una frecuencia de muestreo de 2.048 MS/s donde se obtuvieron parámetros operacionales que se condicen con la teoría para estas condiciones, ganancia de 31.5 dBi para 1 GHz y un piso de ruido de 135 dBm sin utilizar criogenia.\\

Este telescopio cuenta con un rotor motorizado de control remoto con una montura alt azimutal, permitiendo una cobertura completa de la esfera celeste disponible en su ubicación. Se logró una eficiencia de 63\% a 1.42 GHz sin herramientas de construcción sofisticadas ni el uso de materiales costosos para la superficie del reflector parabólico.\\

La detección de la zona de calibración S9 de hidrógeno neutro, significa una puesta en servicio exitosa, pudiendo detectar fuentes astronómicas con una estimación de 100 $\sigma$ aproximadamente para su primera luz, haciendo uso del software de operación remota a través de internet.\\

Se logró la caracterización del patrón de radiación para la frecuencia de interes de CHARTS, obteniendo un HPBW de 7.5 grados centrado a 400 MHz, que a su vez corrobora su capacidad de reconfiguración para otras bandas de observación.\\

Finalmente, el telescopio CPT se convierte en un nuevo recurso clave para nuevas colaboraciones internacionales, destacando el proyecto CHARTS, que busca estudiar las ráfagas rápidas de radio (FRB) y otros fenómenos astrofísicos. En este contexto, se recomienda continuar con la optimización del receptor y su alimentador, para ampliar su rango de frecuencias de operación y fortalecer la infraestructura digital para adquirir mejores observaciones y estudios más complejos.\\

\section{Trabajos Futuros}

Para los trabajos futuros se recomienda la integración de un receptor a base de FPGA, como una RFSoC, para mejorar la capacidad de procesamiento de señales y la adquisición de datos. Esto permitirá la implementación de técnicas de interferometría y la detección de señales de radio más débiles.\\ 

%También se recomienda la integración de un sistema de control centralizado pensado en las capacidades de observación remota.\\

Se debe continuar con el desarrollo de un mejor alimentador, que cubra un mayor ancho de banda, con técnicas de diseño de antenas más avanzadas, para mejorar el desempeño de radiofrecuencia y la disminución de los tiempos de mantenimiento para cambiar de banda de observación de interés.\\

Finalmente, se recomienda la integración de soluciones para mejorar la robustez del sistema y la capacidad de operación en condiciones climáticas adversas, como la instalación de un sistema de enfriamiento activo para la cadena de recepción y métodos de calibración automatizados.\\