\chapter{Conclusiones}

El trabajo de puesta en servicio del telescopio CPT logró cumplir con los objetivos propuestos, destacando la capacidad de observar la línea espectral de hidrógeno neutro, validando su propósito como instrumento de radioastronomía. Las etapas de caracterización permitieron demostrar un rendimiento adecuado en términos de sensibilidad, patrón de radiación y ganancia para un telescopio de estas características. Este éxito inicial da inicio a la vida útil como un nuevo instrumento científico listo para futuras observaciones.\\

Se comisionó un radiotelescopio con tecnologías accesibles y de bajo costo, como las radios definidas por software (SDR). Se pudo desarrollar un sistema de 16348 canales espectrales a una frecuencia de muestreo de 2.048 MS/s. Obteniendo parametros operacionales que se condicen con la teoria para estas condiciones, ganancia de 31.5 dBi para 1 GHz, un piso de ruido de 135dBm sin utilizar criogenia.\\

Con un nuevo radiotelescopio en servicio, se espera poder representar una contribución significativa a las iniciativas científicas y educativas de nuestra comunidad científica y universitaria, ofreciendo un instrumento versátil, para la investigación y en la formación de estudiantes. Además, su diseño adaptable permite que se pueda operar en nuevas bandas de interés y colaborar con otros proyectos con diferentes propósitos científicos.\\

Finalmente, el telescopio CPT se convierte en un nuevo recurso clave para nuevas colaboraciones internacionales, destacando el proyecto CHARTS, que busca estudiar las ráfagas rápidas de radio (FRB) y otros fenómenos astrofísicos. En este contexto, se recomienda continuar con la optimización del receptor y su alimentador, para ampliar su rango de frecuencias de operación y fortalecer la infraestructura digital para adquirir mejores observaciones y estudios más complejos.\\

\section{Trabajos Futuros}

Para los trabajos futuros se recomienda la integración de un receptor a base de FPGA, como una RFSoC, para mejorar la capacidad de procesamiento de señales y la adquisición de datos. Esto permitirá la implementación de técnicas de interferometría y la detección de señales de radio más débiles. También se recomienda la integración de un sistema de control centralizado pensado en las capacidades de observación remota.\\

Se debe continuar con el desarrollo de un mejor alimentador, con técnicas de diseño de antenas más avanzadas, para mejorar el desempeño de radiofrecuencia y la disminución de los tiempos de mantenimiento para cambiar de banda de observación de interés.\\

Finalmente, se recomienda la integración de soluciones para mejorar la robustez del sistema y la capacidad de operación en condiciones climáticas adversas, como la instalación de un sistema de enfriamiento activo para la cadena de recepción y métodos de calibración automatizados.\\