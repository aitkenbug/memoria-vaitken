\begin{appendixs}

\section{Python scripts}

\subsection{cpt\_tracking\_software.py}

Este codigo permite el seguimiento de un objeto en el cielo, utilizando coordenadas altazimutales. El programa se ejecuta en un hilo separado y actualiza la posición del telescopio cada 5 segundos. El usuario puede ingresar comandos para iniciar o detener el seguimiento, cambiar el objetivo o salir del programa.

\begin{sourcecode}[]{python}{}
    import time
    import threading
    from astropy.coordinates import EarthLocation, AltAz, SkyCoord
    from astropy.time import Time
    import spid
    import socket
    from astropy import units as u
    
    # Define your telescope's location
    latitude = -33.395720  # Replace with your latitude in degrees
    longitude = -70.536856  # Replace with your longitude in degrees
    elevation = 868  # Replace with your elevation in meters
    
    # Telescope limits (adjust based on your telescope's safety and design)
    ALT_MIN = 20  # Minimum Altitude in degrees (avoid negative or below horizon)
    ALT_MAX = 150  # Maximum Altitude in degrees
    AZ_MIN = 0   # Minimum Azimuth in degrees
    AZ_MAX = 261 # Maximum Azimuth in degrees
    
    # Global variables for tracking and target
    tracking = False
    current_target = None
    stop_threads = False
    
    def send_angle(alt, az, host='10.17.89.223', port = 23):
        with socket.socket(socket.AF_INET, socket.SOCK_STREAM) as client_socket:
            client_socket.connect((host, port))
            message = spid.encode_command(spid.build_command(round(float(az),1), round(float(alt))))
            client_socket.sendall(message)
    
    
    def get_altz_coords(ra, dec):
        radec = SkyCoord(ra=ra*u.hour, dec=dec*u.deg, frame='icrs')
        # Calan GEOGRAPHIC_COORD
        calan_obs = EarthLocation(lat=-33.3961*u.deg, lon=-70.537*u.deg, height=867*u.m)
        now = Time.now()
        altaz_frame = AltAz(obstime=now,location=calan_obs)
        altaz = radec.transform_to(altaz_frame)
        alt = altaz.alt.deg
        az = altaz.az.deg
    
        if az>200:
            az = 360 -az
            alt = 180 -alt
        return alt, az
    
    
    def is_within_limits(alt, az):
        """Check if the Alt/Az coordinates are within the telescope's limits."""
        return ALT_MIN <= alt <= ALT_MAX and AZ_MIN <= az <= AZ_MAX
    
    
    def track_target():
        """Background thread function to update tracking and send data via socket."""
        global stop_threads
    
        while not stop_threads:
            if tracking and current_target:
                ra, dec = current_target["ra"], current_target["dec"]
                alt, az = get_altz_coords(ra, dec)
    
                if is_within_limits(alt, az):
                    print(f"Tracking {current_target['name']}: Altitude = {alt:.2f}°, Azimuth = {az:.2f}°")
                    data = f"{current_target['name']}: Altitude = {alt:.2f}, Azimuth = {az:.2f}\n"
                    try:
                        send_angle(alt, az)
                    except:
                        print("The controller is not responding")
                else:
                    print(f"{current_target['name']} is outside the telescope limits: Alt = {alt:.2f}°, Az = {az:.2f}°")
            time.sleep(5)  # Update every 5 seconds
    
    
    
    def main():
        global tracking, current_target, stop_threads
    
        print("Welcome to the telescope tracking program!")
        print("Commands: follow, stop, change, exit")
    
        # Start the tracking thread
        tracking_thread = threading.Thread(target=track_target, daemon=True)
        tracking_thread.start()
    
        while True:
            command = input("\nEnter command (follow, stop, change, exit): ").strip().lower()
    
            if command == "follow":
                if current_target:
                    tracking = True
                    print(f"Started following {current_target['name']}...")
                else:
                    print("No target selected. Use 'change' to select a target.")
    
            elif command == "stop":
                tracking = False
                print("Tracking stopped.")
    
            elif command == "change":
                name = input("Enter the name of the object: ").strip()
                ra = float(input("Enter the Right Ascension (RA) in degrees: "))
                dec = float(input("Enter the Declination (Dec) in degrees: "))
    
                current_target = {"name": name, "ra": ra, "dec": dec}
                print(f"Target changed to {name} (RA: {ra}, Dec: {dec}).")
    
            elif command == "exit":
                print("Exiting the program. Goodbye!")
                stop_threads = True
                tracking_thread.join()
                socket_thread.join()
                break
    
            else:
                print("Invalid command. Please try again.")
    
    
    if __name__ == "__main__":
        main()
    
\end{sourcecode}

\subsection{spid.py}

Este script funciona como libreria para codificar y decodificar comandos para el protocolo SPID MD-1 Rot2Prog. Permite convertir ángulos a pulsos, construir comandos de movimiento al stream del socket UDP y decodificar respuestas del controlador.

\begin{sourcecode}[]{python}{}
    # Library to process all bytes formating for the SPID MD-1 Rot2Prog protocol

    deg_res = 2 # 01 - 1, 02 - 0.5, 04 - 0.25, byte - degrees per pulse
    stop_str = "57000000000000000000000F20"
    status_str = "57000000000000000000001F20"
    command_str = "5730303030PH30303030PV2F20"
    restart_str = "57EFBEADDE000000000000EE20"
    
    def angle_to_pulse(angle):
        i = str(int(deg_res * (360 + angle))) # create string of pulses
        if len(i) < 4: # Add the "0" for complet the byte string
            i = "0" + i
        return i
    
    def build_command(az, el):
        command = list(command_str)
        az_pulse = angle_to_pulse(az)
        el_pulse = angle_to_pulse(el)
    
        command[3] = az_pulse[0]
        command[5] = az_pulse[1]
        command[7] = az_pulse[2]
        command[9] = az_pulse[3]
    
        command[10] = "0"
        command[11] = str(deg_res)
    
        command[13] = el_pulse[0]
        command[15] = el_pulse[1]
        command[17] = el_pulse[2]
        command[19] = el_pulse[3]
    
        command[20] = "0"
        command[21] = str(deg_res)
        
        return "".join(command)
    
    def encode_command(msg):
        return bytes.fromhex(msg)
    
    def decode_command(msg):
        response_string = msg.hex()
        H1 = int(response_string[3:4], 16)
        H2 = int(response_string[5:6], 16)
        H3 = int(response_string[7:8], 16)
        H4 = int(response_string[9:10], 16)
        V1 = int(response_string[13:14], 16)
        V2 = int(response_string[15:16], 16)
        V3 = int(response_string[17:18], 16)
        V4 = int(response_string[19:20], 16)
    
        # Calculate angles for Az/El
        az = round((H1 * 100) + (H2 * 10) + H3 + (H4 / 10) -360, 1)
        el = round((V1 * 100) + (V2 * 10) + V3 + (V4 / 10) -360, 1)
    
        return (az, el)    
\end{sourcecode}

\subsection{control.py}

Este script permite la comunicación con el controlador del telescopio a través de un socket UDP. Permite enviar comandos para mover el telescopio, detenerlo y consultar su estad. A su vez tiene los parametros de movimiento que aseguran que cada movimiento no ponga en riesgo los cables con el pedestal. También incluye un hilo para recibir mensajes del controlador y otro para enviar los comandos.

\begin{sourcecode}[]{python}{}
import spid
import socket
import numpy as np
import threading
import sys
import time

movement_monitor = False
stop_threads = False
position = 0,0
stop_position = 0,0

# Function to handle receiving messages from the server
def receive_messages(sock):
    global stop_threads
    global position
    global stop_position

    sock.settimeout(1.0)
    while not stop_threads:
        if stop_threads:
            break
        try:
            message = sock.recv(1024)
            if message:
                position = spid.decode_command(message)
                print(f"Current position -> EL: {position[1]} AZ: {position[0]}")

            else:
                print("Connection closed by the server.")
                stop_threads = True
        
        except socket.timeout:
            continue

        except OSError:
            break

        except ConnectionAbortedError:
            print("Connection was closed.")
            break
        except Exception as e:
            print(f"Error: {str(e)}")
            break

# Function to handle sending messages to the server
def send_messages(sock):
    global stop_threads
    global position
    global stop_position
    global movement_monitor

    while not stop_threads:
        message = input("").split(" ")
        
        if movement_monitor:
            msg = spid.encode_command(spid.status_str)
            if stop_position == position:
                movement_monitor = False

        if message[0] == "stop":
            msg = spid.encode_command(spid.stop_str)
            print("Stoping Telescope Movement")

        if message[0] == "status":
            msg = spid.encode_command(spid.status_str)
            print(f"Actual Position: ")

        if message[0] == "park":
            msg = spid.encode_command(spid.build_command(0,90))
            print("Moving to park position")

        if message[0] == "service":
            msg = spid.encode_command(spid.build_command(0,0))
            print("Moving to service position")

        if message[0] == "restart":
            msg = spid.encode_command(spid.restart_str)

        if message[0] == "move":
            if len(message) < 2:
                print("The command has missing arguments: move [move type] [position] ... ")
            elif message[1] == "elaz":
                if len(message) < 4:
                    print("Elevation and Azimuth movement needs the 2 position arguments")
                else:
                    msg = spid.encode_command(spid.build_command(str_to_f(message[3]),str_to_f(message[2])))
                    stop_pos = message[2], message[3]
                    print(f"Moving to El - Az {message[2]} - {message[3]}")
                    movement_monitor = True

            elif message[1] == "el":
                if len(message) < 3:
                    print("Elevation movement missing argument")
                else:
                    msg = spid.encode_command(spid.build_command(str_to_f(position[0]), str_to_f(message[2])))
                    print(f"Moving to El {message[2]}")
            elif message[1] == "az":
                if len(message) < 3:
                    print("Azimuth movement missing argument")
                else:
                    msg = spid.encode_command(spid.build_command(str_to_f(message[2]), str_to_f(position[1])))
                    print("Moving to Az {message[2]}")

        if message[0].lower() == 'exit':
            print("Closing connection...")
            stop_threads = True
            sock.close()
            break

        try:
            sock.sendall(msg)
        except Exception as e:
            print(f"Error: {str(e)}")
            stop_threads = True
            break
    sys.exit()

def str_to_f(x):
    return round(float(x),1)

def follow_ra_dec(ra,dec):
    pass


def main():
    global stop_threads
    global position
    global stop_position

    # Define server address and port
    host = "10.17.89.223"  # or '127.0.0.1' or server IP address
    port = 23        # Ensure this port matches the server's port

    # Create a socket object
    sock = socket.socket(socket.AF_INET, socket.SOCK_STREAM)

    try:
        # Connect to the server
        sock.connect((host, port))
        print(f"Connected to server {host}:{port}")

        # Read the actual position of the antenna
        sock.sendall(spid.encode_command(spid.status_str))
        time.sleep(0.01)
        angles = spid.decode_command(sock.recv(1024))
        position = angles
        #angles = position
        print(f"Actual Position: Elevation: {position[1]} - Azimuth: {position[0]}")
    
    except Exception as e:
        print(f"Connection error: {str(e)}")
        return

    # Start the receiving thread
    receive_thread = threading.Thread(target=receive_messages, args=(sock,))
    receive_thread.start()

    # Start the sending thread (this is the main thread)
    send_messages(sock)

    # Wait for the receiving thread to finish (if connection closed)
    receive_thread.join()

    sys.exit()

if __name__ == "__main__":
    main()


\end{sourcecode}

\subsection{astrortl.py}

Este codigo contiene funciones para conectarse a un receptor RTL-SDR, configurar sus parámetros y recibir datos. También incluye funciones para procesar la serie muestras de IF y calcular el espectro de frecuencias.

\begin{sourcecode}[]{python}{}
    import socket
    import time
    import numpy as np
    import struct
    import time
    
    def send_command(socket, command, value):
        socket.send(struct.pack('>BI', command, int(value)))
        
    def set_up_rtl(socket, central_freq, sample_rate= 2.048e6, gain_mode=0, gain_index=29, freq_correction = 0):
        #Set up central freq
        send_command(socket, 0x01, central_freq)
        print(f"Central freq set to: {central_freq}")
    
        #Set sample rate
        send_command(socket, 0x02, sample_rate)
        print(f"Sample Rate set to: {sample_rate}")
    
        #Set gain mode
        send_command(socket, 0x03, gain_mode)
        print(f"Gain mode set to: {gain_mode}")
    
        #Set gain by index
        send_command(socket, 0x0d, gain_index)
        print(f"Gain index set to: {gain_index}")
    
        #Set freq correction
        send_command(socket, 0x05, freq_correction)
        print(f"Freq correction set to: {freq_correction}")
    
    def connect_to_rtl(host="127.0.0.1", port=1234):
        sdr_socket = socket.socket(socket.AF_INET, socket.SOCK_STREAM)
        sdr_socket.connect((host, port))
        return sdr_socket
    
    def receive_data(socket, samples=4096):
        buffer_size = samples
        socket.recv(2048)
        raw_data = b''
        
        while len(raw_data) < buffer_size:
            chunk = socket.recv(buffer_size - len(raw_data))
            if not chunk:
                raise RuntimeError("Socket connection closed while receiving data.")
            raw_data += chunk
        #data = np.ctypeslib.as_array(raw_data)
        #iq = raw_data.astype(np.float64).view(np.complex128)
        #iq /= 127.5
        #iq -= (1 + ij)
    
        iq  = [complex(i/(255/2) - 1, q/(255/2) - 1) for i, q in zip(raw_data[::2], raw_data[1::2])]
    
        return iq
    
    def compute_fft(iq_samples, sample_rate):
        fft_size = len(iq_samples)
        spectrum = np.fft.fftshift(np.fft.fft(iq_samples, fft_size))
        spectrum_power = 20 * np.log10(np.abs(spectrum) + 1e-12)  # Convert to dB with numerical stability
        freqs = np.fft.fftshift(np.fft.fftfreq(fft_size, d=1/sample_rate))
        return freqs, spectrum_power
\end{sourcecode}

\subsection{cpt\_rtl\_adquisition.py}

Este script automatiza la observación configurando el receptor y definiendo el tamaño de cada espectro obtenido y a su vez controla el guardado de los espectros en un archivo .npy. El script se ejecuta indefinidamente, guardando los espectros cada 10000 muestras y reiniciando el buffer de datos o hasta que el usuario termine la observación.

\begin{sourcecode}[]{python}{}
import astrortl as rtl
import numpy as np
import matplotlib.pyplot as plt
from collections import deque
import threading
import time

fft_size = 4096
samples = 0
sdr_socket = rtl.connect_to_rtl(host="10.17.89.224", port=1234)
rtl.set_up_rtl(sdr_socket, 1420e6)
file = 0

buffer = []

def data_acquisition(socket, buffer, fft_size):
    freqs, fft = rtl.compute_fft(rtl.receive_data(socket, fft_size), fft_size)
    fft = fft - 80
    buffer = np.append(buffer, fft)
        #time.sleep(0.1)  # Simulate a 10 Hz sampling rate

while True:
    try:
        freqs, fft = rtl.compute_fft(rtl.receive_data(sdr_socket, fft_size), fft_size)
        fft = fft - 80
        buffer.append(fft)
        if samples == 10000-1:
            np.save(f"spectras/milky_way_{file}.npy", np.array(buffer))
            buffer = []
            print(f"Saving buffer {file}")
            file = file + 1

            samples = 0
        else:
            samples = samples + 1
    except:
        sdr_socket.close()
\end{sourcecode}

\subsection{spectra\_avg.py}

\begin{sourcecode}[]{python}{}
    import numpy as np
    import os

    avg_spectra = []
    file = 1
    
    while True:
        files = os.listdir('spectras')
        files = [s for s in files if ".npy" in s]
        try:
            if len(files)>4:
                current = f'spectras/{min(files)}'
                data = np.load(current)
                mean = np.mean(data, axis=0)
                avg_spectra.append(list(mean))
                os.remove(current)
                print(len(avg_spectra))
        except: pass
            
        if len(avg_spectra) == 100:
            np.save(f"spectras/sun_avg_{file}.npy", np.array(avg_spectra))
            avg_spectra = []
            print(f"Saving avg buffer {file}")
            file = file + 1
    
            samples = 0
            
        if len(os.listdir('spectras'))>10:
            print('Too many files')
            break
            

\end{sourcecode}

\subsection{rp\_rtl\_spectrum.py}

Script utilizado para medir el patrón de radiación con la fuente de calibración, incluyendo todas las conversiones de elevación y azimuth para mantener la referencia del patrón



\begin{sourcecode}[]{python}{}
import socket
import time
import numpy as np
import struct
import matplotlib.pyplot as plt
import time
import spid
import decimal
import csv
import pandas as pd
import astrortl

def angle_azimuth_conv(az):
    return 261 - (90 - az)

def angle_elevation_conv(az):
    return (1-2*(abs(az)*np.pi/180)/np.pi)*7

def angle_azimuth_conv_el(el):
    if el < 0:
        return 180 + 171

    else:
        return 171

def angle_elevation_conv_el(el):
    if el < 0:
        
        return 173 + el

    else:
        return el + 7
        

def send_angle(az, host='10.17.89.223', port = 23):
    with socket.socket(socket.AF_INET, socket.SOCK_STREAM) as client_socket:
        client_socket.connect((host, port))
        message = spid.encode_command(spid.build_command(round(float(angle_azimuth_conv(az)),1), round(float(angle_elevation_conv(az)))))
        client_socket.sendall(message)

def send_angle_el(el, host='10.17.89.223', port = 23):
    with socket.socket(socket.AF_INET, socket.SOCK_STREAM) as client_socket:
        client_socket.connect((host, port))
        message = spid.encode_command(spid.build_command(round(float(angle_azimuth_conv_el(el)),1), round(float(angle_elevation_conv_el(el)))))
        client_socket.sendall(message)

\end{sourcecode}

\newpage

\section{Diagramas mecanicos}

\includehfpdf[scale = 0.7 ,pages=-,pagecommand={\subsection{Soporte para tetrápodo}\label{fig:soporte_patas}},linktodoc=true]{draws/soporte_patas}
\includehfpdf[scale = 0.7 ,pages=-,pagecommand={\subsection{Soporte universal para el alimentador}\label{fig:soporte_feed}}]{draws/soporte_feed}
\includehfpdf[scale = 0.7 ,pages=-,pagecommand={\subsection{Soporte para estrella artifical}\label{fig:paleta}}]{draws/paleta}
\includehfpdf[scale = 0.7 ,pages=-,pagecommand={\subsection{Soporte para alimentador de alto ancho de banda}\label{fig:paleta_feed}}]{draws/paleta_feed}
\includehfpdf[scale = 0.7 ,pages=-,pagecommand={\subsection{Soporte para alimentador del dipolo de ARTE}\label{fig:soporte_patch}}]{draws/soporte_patch} 

\end{appendixs}
